\documentclass[14pt,aspectratio=169]{beamer}
\usepackage{etex}
\setbeamertemplate{caption}[numbered]
\setbeamertemplate{caption label separator}{:}
\setbeamercolor{caption name}{fg=normal text.fg}
\usepackage{amssymb,amsmath}
\usepackage{ifxetex,ifluatex}
\usepackage{fixltx2e} % provides \textsubscript
\usepackage{lmodern}
\ifxetex
  \usepackage{fontspec,xltxtra,xunicode}
  \defaultfontfeatures{Mapping=tex-text,Scale=MatchLowercase}
  \newcommand{\euro}{€}
\else
  \ifluatex
    \usepackage{fontspec}
    \defaultfontfeatures{Mapping=tex-text,Scale=MatchLowercase}
    \newcommand{\euro}{€}
  \else
    \usepackage[T1]{fontenc}
    \usepackage[utf8]{inputenc}
      \fi
\fi
% use upquote if available, for straight quotes in verbatim environments
\IfFileExists{upquote.sty}{\usepackage{upquote}}{}
% use microtype if available
\IfFileExists{microtype.sty}{\usepackage{microtype}}{}
\PassOptionsToPackage{hyphens}{url}
\usepackage{hyperref}
\usepackage{ulem}

% Comment these out if you don't want a slide with just the
% part/section/subsection/subsubsection title:
\AtBeginPart{
  \let\insertpartnumber\relax
  \let\partname\relax
  \frame{\partpage}
}
\AtBeginSection{
  \let\insertsectionnumber\relax
  \let\sectionname\relax
  \begin{frame}[plain]
    \tableofcontents[currentsection]
  \end{frame}
}
\AtBeginSubsection{
  \let\insertsubsectionnumber\relax
  \let\subsectionname\relax
  \frame{\subsectionpage}
}

\setlength{\parindent}{0pt}
\setlength{\parskip}{6pt plus 2pt minus 1pt}
\setlength{\emergencystretch}{3em}  % prevent overfull lines
\setcounter{secnumdepth}{0}
% Thanks Richard Darst on how to get a nice Beamer theme.
% See http://rkd.zgib.net/wiki/DebianBeamerThemes

\usepackage{multicol}
\usepackage[absolute,overlay]{textpos}
\usepackage{tikz}
\usepackage{ctable}
\usetikzlibrary{positioning}

\usebackgroundtemplate{\includegraphics[width=\paperwidth]{images/swirl-lightest.pdf}}
\newif\ifplacelogo
\placelogotrue
\logo{\ifplacelogo\includegraphics[viewport=274 335 360 440,width=1cm]{images/openlogo-nd.pdf}\fi}

\definecolor{debianred}{rgb}{.780,.000,.211} % 199,0,54
\definecolor{debianblue}{rgb}{0,.208,.780} % 0,53,199
\definecolor{debianlightbackgroundblue}{rgb}{.941,.941,.957} % 240,240,244
\definecolor{debianbackgroundblue}{rgb}{.776,.784,.878} % 198,200,224

\usetheme{Boadilla}
\setbeamertemplate{navigation symbols}{}

\usecolortheme[named=debianbackgroundblue]{structure}
\setbeamercolor{normal text}{fg=black}
\setbeamercolor{titlelike}{fg=debianblue}
\setbeamercolor{sidebar}{fg=debianred,bg=debianbackgroundblue}

\setbeamercolor{palette sidebar primary}{fg=debianred}
\setbeamercolor{palette sidebar secondary}{fg=debianred}
\setbeamercolor{palette sidebar tertiary}{fg=debianred}
\setbeamercolor{palette sidebar quaternary}{fg=debianred}

\setbeamercolor{section in toc}{fg=debianred}
\setbeamercolor{subsection in toc}{parent=debianred}

\setbeamercolor{item}{fg=debianred}

\setbeamercolor{block title}{fg=debianblue}


\title[Reproducible buster]{Reproducible
buster}
%\subtitle{We've made lots of progress, \\
%but we are still far from our goals \\
%of changing the (software) world.}
\author[h01ger,jathan,lamby,stevenc]{%
   \texorpdfstring{
            \centering
            Holger 'h01ger' Levsen
   }{h01ger} \\
   \texorpdfstring{
            \centering
            Jonathan Bustillos Osornio
   }{jathan} \\
   \texorpdfstring{
            \centering
            Chris Lamb
   }{lamby} \\
   \texorpdfstring{
            \centering
            Steven Chamberlain
   }{stevenc}
}
\date[OSSE]{%
 MiniDebConf 2018 in Hamburg\\
 \small{2018-05-20}}

\begin{document}
\placelogofalse

\begin{frame}[plain]
 \titlepage
\end{frame}

\begin{frame}
 \frametitle{Debian reproducible builds contributors}
 \begin{center}
  \begin{columns}
   \footnotesize
   \column{.30\linewidth}
    {akira} \\
    {Alexis Bienvenüe} \\
    {Andrew Ayer} \\
    {Asheesh Laroia} \\
    Boyuan Yang \\
    {Ceridwen} \\
    {Chris Lamb} \\
    {Chris West} \\
    {Christoph Berg} \\
    Clint Adams \\
    Dafydd Harries \\
    {Daniel Kahn Gillmor} \\
    {Daniel Shahaf} \\
    Daniel Stender \\
    David Suarez \\
    {Dhole} \\
    Drew Fisher \\
    Emmanuel Bourg \\
    Emanuel Bronshtein \\
    \column{.30\linewidth}
    Esa Peuha \\
    {Fabian Wolff} \\
    {Guillem Jover} \\
    Hans-Christoph Steiner \\
    Harlan Lieberman-Berg \\
    {Helmut Grohne} \\
    \only<1>{Holger Levsen}\only<2>{{\color{debianred} Holger Levsen}} \\
    HW42 \\
    Intrigeri \\
    {Jelmer Vernooij} \\
    {josch} \\
    Juan Picca \\
    Juliana Rodrigues \\
    {Lunar} \\
    Maria Glukhova \\
    Mathieu Bridon \\
    {Mattia Rizzolo} \\
    Nicolas Boulenguez \\
    {Niels Thykier} \\
    Niko Tyni \\
   \column{.30\linewidth}
    {Paul Wise} \\
    Peter De Wachter \\
    Philip Rinn \\
    {Reiner Herrmann} \\
    Robbie Harwood \\
    {Santiago Vila} \\
    {Sascha Steinbiss} \\
    {Satyam Zode} \\
    {Scarlett Clark} \\
    {Stefano Rivera} \\
    {Stéphane Glondu} \\
    {Steven Chamberlain} \\
    Tom Fitzhenry \\
    Vagrant Cascadian
    {Valerie Young} \\
    Valentin Lorentz \\
    {Wookey} \\
    {Ximin Luo} \\
  \end{columns}
 \end{center}
\end{frame}


\placelogofalse

\begin{frame}
 \frametitle{The solution: Reproducible Builds}

 \begin{center}
 \Large{
 Promise that \textbf{anyone} can \textbf{always} and \textbf{independently} generate
 bit by bit identical binary packages from a given source}
\end{center}
\end{frame}


\begin{frame}[plain]
\begin{center}
 \Huge{This should become the \textbf{norm}.}

 \visible<2>{\small{ We want to change the meaning of "free software":

  it's only free software if it's reproducible!}}
\end{center}
\end{frame}

\begin{frame}[fragile]
 \frametitle{An important link in the chain of secure software distribution}
 \begin{itemize}
  \item But it's just one link in a big chain.
  \item Software security consists of more than secure distribution. I won't go into detail here, but think software lifecycle management.
  \item<2> It's a critical link though, linking sources to binaries and vice versa. The problem with randomness is, that one can never be sure. The problem with reproducible builds is, that it's a lot of efford to prove something, which we used to believe and take for granted.
  \item<2> Reproducible Builds are just one link in a big chain, turning that chain into a foundation to build trust upon. There are many ways to subvert this trust, but that's no excuse to stay a believer.
 \end{itemize}
\end{frame}


\begin{frame}[fragile]
 \frametitle{More benefits than "just" security…}
 \begin{itemize}
  \item Lots and lots of QA benefits - we've found so many subtile bugs.
  \item<2-5> Google does reproducible builds, to save time and money.
  \item<3-5> Smaller deltas, thus faster updates possible (for packages and
  images).
  \item<4-5> Side effect: meaningful binary diff between two versions.
  \item<5> …
 \end{itemize}
\end{frame}


\begin{frame}
 \frametitle{Some milestones in history}
 \begin{itemize}
\item<7> Cygnus.com (1992)
\item Bitcoin (2011)
\item<2-7> Tor (2013)
\item<3-7> Debian (2013, but really in 2014, and 2017 in \texttt{debian-policy})
\item<4-7> Core Infrastructure Iniative (2014)
\item<5-7> FreeBSD, coreboot, LEDE, openSUSE, NetBSD (2016++)
\item<6-7> Tails (2017)
 \end{itemize}
\end{frame}

\placelogotrue

\begin{frame}
 \frametitle{2017: Stretch at 94\% and in \texttt{debian-policy}}
 \begin{tikzpicture}[remember picture]
  \node[shift={(-0.5\paperwidth, \paperheight)},at=(current page.south east)] {
    \includegraphics[height=0.65\paperheight]{images/stats_pkg_state_testing.png}
  };
 \end{tikzpicture}
 \begin{center}
  \footnotesize{23,344 (94.0\%) out of 24,821 source packages are reproducible \\
    in our test framework on \texttt{amd64}}
  \vfill
 \end{center}
\end{frame}

\placelogotrue

\begin{frame}
 \frametitle{"Misleading success"}
 \begin{itemize}
	 \item<2-4> In Debian in 2017 Reproducible Builds went into \texttt{debian-policy} and hopefully by 2019 we'll have \textbf{some} infrastructure and \textbf{some} user tools in Debian \texttt{stable}. Even reaching 100\% Reproducible Builds by 2021 is ambitious. 6\% is a lot if you're talking about 25000 packages but people seem to forget this.
	 \item<3-4> Despite the Debian developer community strongly supporting this, progress is difficult: it really get's complicated again on the last miles. (Think: 6\%, infrastructure \& user tools.)
	 \item<4> I might be wrong, I hope I am, but I only know of two other ("big or relevant", sorry) projects with similar commitment: Tails and Tor. But for them, a small How-To is sufficient.
 \end{itemize}
\end{frame}

\begin{frame}
 \frametitle{"Misleading success, cont."}
 \begin{itemize}
 \item To sum this up: we are at 94\% of \textbf{theoretically} being able to do Reproducible Builds. Which means: the software supports it, in theory! What's lacking is infrastructure (think distribution of all those hashes) and user tools, so users can benefit from it.
 \item<2-4> Debian is the most advanced (big distro) here. The others haven't even started.
 \item<3-4> We need to keep doing what we have been doing (and which I'm going to explain in more detail now) and we need to do more and new things. And we need \textbf{you} to join this efford, especially if you are working on some other project than Debian!
 \item<4> First 90\% take 90\% of the time \& last 9\% take another 90\%…

 \end{itemize}
\end{frame}

\section{Common ressources}

{
\usebackgroundtemplate{%
 \begin{tikzpicture}[remember picture,overlay]%
  \node[shift={(-0.1\paperwidth, 0.15\paperheight)},at=(current page.south east)] {
    \includegraphics[width=0.2\paperwidth]{images/diffoscope_logo.png}
  };
 \end{tikzpicture}%
}

\begin{frame}{diffoscope}
 \frametitle{Debugging problems: \texttt{https://try.diffoscope.org}}

 \begin{itemize}
  \item Examines differences \textbf{in depth}.
  \item Recursively unpacks archives, uncompresses PDFs, disassembles
  binaries, unpacks Gettext files, …
  \item Easy to extend to new file formats.
  \item Falls back to binary comparison.
  \item Outputs HTML or plain text with human readable differences.
  \item Available from \texttt{git}, PyPI, Debian, \\
   Arch Linux, Guix, Homebrew, Fedora. Works on BSD.
  \item Maintainers in other distros wanted.
  \item \url{https://diffoscope.org/}
 \end{itemize}
\end{frame}


\begin{frame}
 \frametitle{\texttt{diffoscope} example (HTML output)}
 \begin{tikzpicture}[remember picture]
  \node[at=(current page.center)] {
   \includegraphics[width=0.9\paperwidth]{images/diffoscope_example_html.png}
  };
 \end{tikzpicture}
\end{frame}


\begin{frame}
 \frametitle{\texttt{diffoscope} is "just" for debugging}

 \begin{itemize}
  \item Reminder: \texttt{diffoscope} is for \textbf{debugging}
  \item "reproducible" according to our definition means: \textbf{bit by bit
  identical}. So the tools for testing whether something is reproducible are
  either \texttt{diff} or \texttt{sha256sum}!
  \item<2> \texttt{https://try.diffoscope.org}
 \end{itemize}
\end{frame}

}



\placelogofalse


\begin{frame}
 \frametitle{tests.reproducible-builds.org}

 \begin{itemize}
  \item Continuously testing Debian \texttt{testing}, \texttt{unstable} and
  \texttt{experimental} 
  \item Also testing: coreboot, LEDE, NetBSD, FreeBSD and F-Droid.
  Unmaintained tests: Arch Linux, Fedora and OpenWrt.
  \item 51 nodes (amd64/i386/arm64/armhf), 500 cores and 1 TB RAM
  \item 513 jenkins jobs running on jenkins.debian.net
  \item 51 scripts in Python and Bash, 357 lines of code in average
  \item 39 contributors for \texttt{jenkins.debian.net.git}
 \end{itemize}
 \begin{center}
  \includegraphics[height=0.1\paperheight]{images/profitbricks_logo.png}
  \hspace{0.1\paperwidth}
  \includegraphics[height=0.1\paperheight]{images/profitbricks_logo.png}
  \hspace{0.1\paperwidth}
  \includegraphics[height=0.1\paperheight]{images/codethink.png}
  \hspace{0.1\paperwidth}
 \includegraphics[height=0.1\paperheight]{images/debian_logo.png}
  \hspace{0.1\paperwidth}
 \end{center}
\end{frame}

\placelogotrue

\begin{frame}[fragile]
 \frametitle{Variations (when testing Debian)}

 \begin{center}
  \begin{table}
   \resizebox{0.95\textwidth}{!}{%
    \begin{tabular}{l|ll}
\textbf{variation} & \textbf{first build} & \textbf{second build} \\
\hline
hostname & \texttt{jenkins} & \texttt{i-capture-the-hostname} \\
domainname & \texttt{debian.net} & \texttt{i-capture-the-domainname} \\
\texttt{env TZ} & \texttt{GMT+12} & \texttt{GMT-14} \\
\texttt{env LANG} & \texttt{C} & \texttt{fr\_CH.UTF-8} \\
\texttt{env LC\_ALL} & not set & \texttt{fr\_CH.UTF-8} \\
\texttt{env USER} & \texttt{pbuilder1} & \texttt{pbuilder2} \\
uid & \texttt{1111} & \texttt{2222} \\
gid & \texttt{1111} & \texttt{2222} \\
UTS namespace & shared with the host & \textit{modified using \texttt{/usr/bin/unshare --uts}} \\
kernel version & Linux 3.16 or 4.X & on amd64 always varied, on armhf
sometimes \\
umask & 0022 & 0002 \\
CPU type & \multicolumn{2}{l}{varied on i386} \\
 & on armhf varied a bit, not on amd64 & \\
filesystem & \multicolumn{2}{l}{same for both builds on amd64: (\texttt{tmpfs}), on armhf \texttt{ext3/4}} \\
 & & \textit{(and we have} \texttt{disorderfs}\textit{, but the code is disabled)} \\
year, month, date & \multicolumn{2}{l}{on amd64: 398 days variation, on armhf not yet} \\
hour, minute & \multicolumn{2}{l}{hour is usually the same… usually, the minute differs… } \\
\textit{everything else} & \multicolumn{2}{l}{\textit{is likely the same…}}
    \end{tabular}
   }
  \end{table}
 \end{center}
\end{frame}

\section{Status Debian}

\begin{frame}
 \frametitle{Status golang}

 \begin{itemize}
  \item<2-4> Golang binaries are bit by bit reproducible!
  \item<3-4> Except when the build path is varied…
  \item<4> \url{https://anonscm.debian.org/cgit/pkg-golang/golang.git/tree/debian/patches/0002-reproducible-BUILD\_PATH\_PREFIX\_MAP.patch?h=golang-1.9}

 \end{itemize}
 \begin{center}
  \includegraphics[height=0.2\paperheight]{images/golang.png}
 \end{center}
\end{frame}

\begin{frame}
 \frametitle{\texttt{BUILD\_PATH\_PREFIX\_MAP}}
 \begin{itemize}
  \item "This specification describes the environment variable BUILD\_PATH\_PREFIX\_MAP for build tools to exchange information about the build-time filesystem layout, to generate reproducible output where all embedded paths are independent of that layout."
  \item \url{https://reproducible-builds.org/specs/build-path-prefix-map/}
  \item<2-3> Build path variations only enabled when testing Debian \texttt{unstable}.
  \item<3> The easy workaround today is to rebuild in the same path. But it is a workaround, why should the path be embedded in the binary?
 \end{itemize}
\end{frame}


\placelogotrue



\begin{frame}
 \frametitle{Progress in Debian \texttt{unstable}}
 \begin{tikzpicture}[remember picture]
  \node[shift={(-0.5\paperwidth, \paperheight)},at=(current page.south east)] {
    \includegraphics[height=0.65\paperheight]{images/stats_pkg_state_unstable.png}
  };
 \end{tikzpicture}
 \begin{center}
  \footnotesize{23,243 (86.2\%) out of 26,957 source packages are reproducible \\
    in our test framework on \texttt{amd64}} (difference due to build path variations)
  \vfill
 \end{center}
\end{frame}

\begin{frame}
 \frametitle{some more details on tests.reproducible-builds.org}

 \begin{itemize}
  \item \url{https://reproducible.debian.net/$src}
  \item<2> 49 package sets 
 \end{itemize}
\end{frame}

\begin{frame}
	\frametitle{Debian notes \& issues on tests.reproducible-builds.org}

 \begin{itemize}
  \item 283 categorised distinct issues
  \item 6,220 notes
  \item 1,594 unreproducible packages in \texttt{buster/amd64} (testing), but only
  263 without a note (3,433 in \texttt{unstable} but also only 378 without a
  note)
  \item maintained in \texttt{notes.git} by 58 contributors
  \item currently Debian only, but cross distro notes are planned
 \end{itemize}
\end{frame}


\begin{frame}
 \frametitle{Debian \texttt{.buildinfo} files}

 \begin{itemize}
  \item Aggregates in the same file:
   \begin{itemize}
    \item Sources (checksums)
    \item Generated binaries (checksums)
    \item Packages used to build (with specific version, checksums coming soon)
   \end{itemize}
  \item Can be later used to exactly recreate environment
  \item For Debian, all versions are available from \url{snapshot.debian.org}
  \item<2>Concept is universial, there are some rough draft implementations elsewhere, but nothing proven nor tested.
 \end{itemize}
\end{frame}


\begin{frame}
 \frametitle{Progress in the Debian bug tracker}
 \begin{tikzpicture}[remember picture]
  \node[shift={(-0.5\paperwidth, \paperheight)},at=(current page.south east)] {
    \includegraphics[height=0.65\paperheight]{images/stats_bugs_sin_ftbfs_state.png}
  };
 \end{tikzpicture}
 \begin{center}
  \footnotesize{As a rule, we file bugs with patches. \\
  There are very few exceptions.}
  \vfill
 \end{center}
\end{frame}



\begin{frame}
	\frametitle{Debian summary - situation in Stretch (Debian 9)}
 \begin{itemize}
   \item All our required changes to build reproducible packages are included in Stretch, released in June 2017!
 \item<2-4> This is/was a proof-of-concept, Debian is neither 94\% reproducible nor
  86\%. (and 10\% = 2,500 sources packages!)
  \item<3-4> 94\% of the source packages in Stretch can build reproducible packages. But less than 20\% of the released binaries are reproducible…
  \item<3-4> Because, Debian does not (yet?) do full rebuilds before
  releasing… so stuff is in the archive which is not reproducible unless it's
  rebuild.
  \item<4> And then we don't distribute \texttt{.buildinfo} files yet.
   That (and user tools) still needs more \it{design} and code.
 \end{itemize}
\end{frame}


\begin{frame}
	\frametitle{Debian summary - situation for derivates \& the future}
 \begin{itemize}
  \item Stretch's source code is 94\% reproducible and all required change are included.
  \item So others, eg Canonical can take our work now and make Ubuntu 17.10
  (partly) reproducible…
  \item<2-4> Debian 10, "buster", will be partly reproducible in 2019.
  \item<3-4> Since August 2017 \texttt{debian-policy} mandates that packages \textbf{should} be reproducible.
  \item<4> We hope \texttt{debian-policy} will mandate 100\%
	  reproducible builds ("\textbf{must}") for Debian 11, "bullseye", in 2021.
  \item<4> And even then, there can be exceptions…
 \end{itemize}
\end{frame}

\begin{frame}
 \frametitle{Tell the world \& collaborate}

 \begin{itemize}
  \item "We don't care about Debian (only), we care about free and open
	  source software." (By now this is pretty obvious.)
  \item<2-4> 130 Weekly reports since May 2015
  \item<3-4> First Reproducible World Summit in December 2015 (Athens, Greece)
   \begin{itemize}
    \item<3-4> \texttt{reproducible.debian.net} has become \texttt{tests.reproducible-builds.org}
   \end{itemize}
    \item<3-4> Second Reproducible World Summit in December 2016 in Berlin
    \item<3-4> Third Reproducible World Summit at the End of October 2017 in Berlin - contact me if you want to attend.
   \item<4> GSoC and Outreachy
 \end{itemize}
\end{frame}



\section{Future work}

\begin{frame}
 \frametitle{Future work}
 \begin{itemize}
	 \item<1-3> So far we mostly worked on making reproducible builds possible ("in theory")… and we need to keep doing this until we reached 100\%.
 \item<2-3> We'll need constant tests for future code. So we need to keep our tests running, forever? And we need external rebuilders too.

 \item<3> And then, this still needs tools, infrastructure and policies to become
 meaningful and to be used in practice.
 \end{itemize}
\end{frame}

\begin{frame}
 \frametitle{Rebuilds and sharing signed checksums}
 \begin{itemize}
  \item Almost no work has been done here yet. We mostly were busy with the first step:
  being able to rebuild reproducibly…
  \item Different projects, different solutions:
 \begin{itemize}
  \item<2-3> something like \texttt{.buildinfo} files (defining the environment,
  the input and the output(s)) will be needed everywhere:
 \begin{itemize}
  \item<2-3> implemented for Debian (both in sbuild and well as
  buildinfo.debian.net) but not ready nor really usable yet.
  \item<2-3> some work started for coreboot, LEDE/OpenWrt, Arch linux and Fedora (mock/koji), openSUSE (OpenBuildService) and Guix/NixOS.
 \end{itemize}
 \item<3> Still needs work: storage and distribution of .buildinfo files.
 \item<3> Still needs work: rebuilders and external signers.
 \end{itemize}
 \end{itemize}
\end{frame}

\begin{frame}
 \frametitle{Rebuilders and sharing signed checksums, cont.}
 \begin{itemize}
  \item Individually signed checksums (think web of trust) could work in the
  Debian case (we have a gpg web of trust), but IMO won't scale.
  \item { Another idea: rebuilders, run by large organisations
  (ACLU, CCC, Deutsche Bank, Greenpeace, NASA, NSA, US-Army).}
  \item Fedora rebuilds Debian, Debian rebuilds openSUSE, openSUSE rebuilds
  NetBSD, etc…
  \item Big customers could just rebuild everything themselves and compare that to offical builds (and hopefully share their results).
 \end{itemize}
\end{frame}


\begin{frame}
 \frametitle{Integration in user tools}
 \begin{itemize}
  \item "Do you really want to install this unreproducible software (y/N)"
  \item<2-3> "Do you want to build those packages which have unconfirmed checksums,
  before installing? (Y/n)"
  \item<3>{ "How many signed checksums do you require to call a package
  'reproducible'?" - and whom do you trust?}
 \end{itemize}
\end{frame}


\section{Questions, comments, ideas?}

\placelogofalse

\begin{frame}
 \frametitle{Questions, comments, ideas?}

 \begin{itemize}
  \item \url{https://reproducible-builds.org/}
  \item \texttt{\#reproducible-builds} on \texttt{irc.OFTC.net}
  \item \url{https://lists.reproducible-builds.org}
  \item twitter: @ReproBuild
  \item<2> Mike and Seth's talk from 31c3 about motivations
  \item<2> Lunar's talk about fixing reproducible issues from CCCamp 15
  \end{itemize}
\end{frame}

\placelogotrue

\begin{frame}
 \frametitle{Thanks to…! …and thank \textbf{you}, too!}

 \begin{itemize}
  \item
    {All “Reproducible Builds” contributors \\
        {\small (you are just \textbf{so} awesome!)}}
\end{itemize}

 \begin{center}
  \includegraphics[height=0.1\paperheight]{images/linux_foundation_logo.png}
  \hspace{0.1\paperwidth}
  \includegraphics[height=0.1\paperheight]{images/cii_logo.png}
 \end{center}

\end{frame}


\placelogotrue

\begin{frame}{}
\begin{textblock}{12}(2, 6)
    \tiny{
      Copyright \copyright{} 2014--2017 \\
         Holger Levsen \texttt{holger@layer-acht.org} and others.\\[3.0mm]
      Copyright of images included in this document are held by
      their respective owners.
      \\[3.0mm]
      This work is licensed under the \alert{Creative Commons
        Attribution-Share Alike 3.0} License.  To view a copy of this
      license, visit
      \url{http://creativecommons.org/licenses/by-sa/3.0/} or send a
      letter to Creative Commons, 171 Second Street, Suite 300, San
      Francisco, California, 94105, USA.
      \\[2.0mm]
      % Give a link to the 'Transparent Copy', as per Section 3 of the GFDL.
      The source of this document is available from
      \url{https://anonscm.debian.org/git/reproducible/presentations.git}.
    }
  \end{textblock}
\end{frame}

\end{document}
