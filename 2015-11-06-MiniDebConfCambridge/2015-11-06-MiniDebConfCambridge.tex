\documentclass[14pt]{beamer}
\setbeamertemplate{caption}[numbered]
\setbeamertemplate{caption label separator}{:}
\setbeamercolor{caption name}{fg=normal text.fg}
\usepackage{amssymb,amsmath}
\usepackage{ifxetex,ifluatex}
\usepackage{fixltx2e} % provides \textsubscript
\usepackage{lmodern}
\ifxetex
  \usepackage{fontspec,xltxtra,xunicode}
  \defaultfontfeatures{Mapping=tex-text,Scale=MatchLowercase}
  \newcommand{\euro}{€}
\else
  \ifluatex
    \usepackage{fontspec}
    \defaultfontfeatures{Mapping=tex-text,Scale=MatchLowercase}
    \newcommand{\euro}{€}
  \else
    \usepackage[T1]{fontenc}
    \usepackage[utf8]{inputenc}
      \fi
\fi
% use upquote if available, for straight quotes in verbatim environments
\IfFileExists{upquote.sty}{\usepackage{upquote}}{}
% use microtype if available
\IfFileExists{microtype.sty}{\usepackage{microtype}}{}
\PassOptionsToPackage{hyphens}{url}
\usepackage{hyperref}
\usepackage{ulem}

% Comment these out if you don't want a slide with just the
% part/section/subsection/subsubsection title:
\AtBeginPart{
  \let\insertpartnumber\relax
  \let\partname\relax
  \frame{\partpage}
}
\AtBeginSection{
  \let\insertsectionnumber\relax
  \let\sectionname\relax
  \begin{frame}[plain]
    \tableofcontents[currentsection]
  \end{frame}
}
\AtBeginSubsection{
  \let\insertsubsectionnumber\relax
  \let\subsectionname\relax
  \frame{\subsectionpage}
}

\setlength{\parindent}{0pt}
\setlength{\parskip}{6pt plus 2pt minus 1pt}
\setlength{\emergencystretch}{3em}  % prevent overfull lines
\setcounter{secnumdepth}{0}
% Thanks Richard Darst on how to get a nice Beamer theme.
% See http://rkd.zgib.net/wiki/DebianBeamerThemes

\usepackage{ctable}
\usepackage{multicol}
\usepackage{tikz}
\usetikzlibrary{positioning}

\usebackgroundtemplate{\includegraphics[width=\paperwidth]{images/swirl-lightest.pdf}}
\logo{\includegraphics[viewport=274 335 360 440,width=1cm]{images/openlogo-nd.pdf}}

\definecolor{debianred}{rgb}{.780,.000,.211} % 199,0,54
\definecolor{debianblue}{rgb}{0,.208,.780} % 0,53,199
\definecolor{debianlightbackgroundblue}{rgb}{.941,.941,.957} % 240,240,244
\definecolor{debianbackgroundblue}{rgb}{.776,.784,.878} % 198,200,224

\usetheme{Boadilla}
\setbeamertemplate{navigation symbols}{}

\usecolortheme[named=debianbackgroundblue]{structure}
\setbeamercolor{normal text}{fg=black}
\setbeamercolor{titlelike}{fg=debianblue}
\setbeamercolor{sidebar}{fg=debianred,bg=debianbackgroundblue}

\setbeamercolor{palette sidebar primary}{fg=debianred}
\setbeamercolor{palette sidebar secondary}{fg=debianred}
\setbeamercolor{palette sidebar tertiary}{fg=debianred}
\setbeamercolor{palette sidebar quaternary}{fg=debianred}

\setbeamercolor{section in toc}{fg=debianred}
\setbeamercolor{subsection in toc}{parent=debianred}

\setbeamercolor{item}{fg=debianred}

\setbeamercolor{block title}{fg=debianblue}

\title[Reproducible builds]{Beyond reproducible builds}
\subtitle{we are not there yet and "there" is only the first third}
\author[lamby \& h01ger]{%
   \texorpdfstring{
        \begin{columns}
            \column{.45\linewidth}
            \centering
            Chris 'lamby' Lamb \\
            \href{mailto:lamby@debian.org}{lamby@debian.org}
            \column{.45\linewidth}
            \centering
            Holger 'h01ger' Levsen\\
            \href{mailto:holger@debian.org}{holger@debian.org}
        \end{columns}
   }{lamby \& h01ger}}
\institute[Debian]{}
\date[Mini-DebConf Cambridge 2015]{%
 Mini DebConf 2015,\\
 Cambridge, UK\\
 \small
 2015-11-06}

\begin{document}

\begin{frame}
 \titlepage
\end{frame}


\begin{frame}
 \frametitle{Debian reproducible builds team}
 \begin{center}
  \begin{columns}
   \small
   \column{.33\linewidth}
    {akira} \\
    {Andrew Ayer} \\
    {Asheesh Laroia} \\
    \only<1>{Chris Lamb}\only<2>{{\color{debianblue} Chris Lamb}} \\
    {Chris West} \\
    {Christoph Berg} \\
    {Daniel Kahn Gillmor} \\
    David Suarez \\
    {Dhole} \\
    Drew Fisher \\
    Esa Peuha \\
    {Guillem Jover} \\
   \column{.33\linewidth}
    Hans-Christoph Steiner \\
    {Helmut Grohne} \\
    \only<1>{Holger Levsen}\only<2>{{\color{debianblue} Holger Levsen}} \\
    Jelmer Vernooij \\
    {josch} \\
    Juan Picca \\
    {Lunar} \\
    Mathieu Bridon \\
    {Mattia Rizzolo} \\
    Nicolas Boulenguez \\
    {Niels Thykier} \\
    Niko Tyni \\
   \column{.33\linewidth}
    {Paul Wise} \\
    Peter De Wachter \\
    Philip Rinn \\
    {Reiner Herrmann} \\
    {Stefano Rivera} \\
    {Stéphane Glondu} \\
    {Steven Chamberlain} \\
    Tom Fitzhenry \\
    Valentin Lorentz \\
    {Wookey} \\
    {Ximin Luo} \\
  \end{columns}
 \end{center}
\end{frame}

\begin{frame}
 \frametitle{Who are you?}
  \only<2-4>{Who is…} \\
  \only<3-4>{Who is…} \\
  \only<4>{Who…} \\
\end{frame}

\section{Status}

\begin{frame}[fragile]
 \frametitle{In depth explaination of the problem}

 \begin{center}
  \includegraphics[width=0.7\textwidth]{images/31c3.png}

  Available on \url{media.ccc.de}, 31c3
 \end{center}
\end{frame}

\begin{frame}
 \frametitle{The solution}

 \begin{center}
 \Large
 enable anyone to reproduce\\
 identical binary packages\\
 from a given source
\end{center}

\end{frame}

\begin{frame}
 \frametitle{The solution}

 \begin{center}
 We call this:

 \Huge
 “reproducible builds”
 \end{center}
\end{frame}

\begin{frame}
 \frametitle{So trendy!}

 \begin{itemize}
 \item Bitcoin (\textbf{"done"})
 \item Tor (\textbf{"done"})
 \item Debian (\emph{in progress})
 \item Coreboot (\textbf{"done"})
 \item OpenWrt (\emph{in progress})
 \item NetBSD (\emph{in progress})
 \item FreeBSD (\emph{in progress})
 \item Arch Linux (\emph{in progress})
 \item \ldots{}
\end{itemize}

\end{frame}


\begin{frame}[plain]
 \begin{tikzpicture}[remember picture,overlay]
  \node[at=(current page.center)] {
    \includegraphics[width=\paperwidth]{images/wholeworld.jpg}
    % Credits to Kevin ‘Chuise’ Jackson
    % http://dumhi.com/about
  };
 \end{tikzpicture}
\end{frame}

\begin{frame}[plain]
\begin{center}
 \Huge It should become the \textbf{norm}.\\
 \only<2>{\small We want to change the meaning of "free software": \\
  it's only free software if it is reproducible!}
\end{center}

\end{frame}

\section{Current status}

\begin{frame}[plain]
 \frametitle{Progress in Debian \texttt{unstable}}
 \begin{center}
  \includegraphics[width=\paperwidth]{images/stats_pkg_state.png}
  
  19399 out of 23033 source packages are reproducible in our current
  test framework.
 \end{center}
\end{frame}

\begin{frame}
 \frametitle{What we did since summer 2014}

 \begin{itemize}
  \item Agreed on using a fixed build path: \texttt{/build}
  \item Record the build environment: \texttt{.buildinfo}
  \item \texttt{strip-nondeterminism}
  \item \texttt{reproducible.debian.net}
  \item \texttt{diffoscope} (formerly known as \texttt{debbindiff})
  \item \texttt{SOURCE\_DATE\_EPOCH}
  \item \texttt{disorderfs}
  \item Many many patches: \texttt{dpkg}, \texttt{debhelper}, \texttt{cdbs}, \texttt{sbuild}, …
  \item\only<2>{Tell the world \& collaborate}
 \end{itemize}
\end{frame}


\begin{frame}
 \frametitle{Tell the world \& collaborate}

 \begin{itemize}
  \item Two recent talks:
   \begin{itemize}
    \item 2015-08-13: Chaos Communication Camp 2015
    \item 2015-08-20: DebConf15
    \item (both have subtitles!)
   \end{itemize}
  \item Linked on the wiki:
    {\small \url{https://wiki.debian.org/ReproducibleBuilds/About#Presentations}}
  \item Weekly reports since May 2015
 \end{itemize}
\end{frame}

\begin{frame}
 \frametitle{Tell the world \& collaborate, cont.}

 \begin{itemize}
  \item Summit in December 2015 in Athens
   \begin{itemize}
    \item 40 people from 16 projects
   \end{itemize}
  \item \texttt{https://reproducible-builds.org}
  \begin{center}
   \includegraphics[width=0.6\textwidth]{images/rbwww1.png}
  \end{center}
 \end{itemize}
\end{frame}



\begin{frame}
 \frametitle{update on reproducible.debian.net}

 \begin{itemize}
  \item maintained in \texttt{jenkins.debian.net.git}, 27 contributors
  \item 4k lines of Python and 5k lines Bash code
  \item 111 jenkins jobs now running on 10 hosts
  \item Continuously testing Debian testing, unstable and experimental
  \begin{itemize}
   \item Previously amd64 only, now also armhf, more to come…
  \end{itemize}
  \item Not just testing Debian, but also Coreboot, OpenWrt, NetBSD, FreeBSD,
  Archlinux and soon Fedora
  \item 109 cores and 194 GB Ram split on 8 amd64 VMs, and 12 cores and 6 GB ram one 4 armhf nodes, provided by vagrant.
  \item we more more more arm(64) cores! (but not this small...)
  \item Thanks to ProfitBricks for providing amd64 servers:
 \end{itemize}
 \vfill
 \begin{center}
 \includegraphics[height=0.15\paperheight]{images/profitbricks_logo.png}
 \end{center}
\end{frame}

\begin{frame}[fragile]
 \frametitle{Variations on reproducible.debian.net}

 \begin{center}
  \begin{table}
   \resizebox{0.95\textwidth}{!}{%
    \begin{tabular}{l|ll}
\textbf{variation} & \textbf{first build} & \textbf{second build} \\
\hline
hostname & \texttt{jenkins} & \texttt{i-capture-the-hostname} \\
domainname & \texttt{debian.net} & \texttt{i-capture-the-domainname} \\
\texttt{env TZ} & \texttt{GMT+12} & \texttt{GMT-14} \\
\texttt{env LANG} & \texttt{en\_GB.UTF-8} & \texttt{fr\_CH.UTF-8} \\
\texttt{env LC\_ALL} & not set & \texttt{fr\_CH.UTF-8} \\
\texttt{env USER} & \texttt{pbuilder1} & \texttt{pbuilder2} \\
uid & \texttt{1111} & \texttt{2222} \\
gid & \texttt{1111} & \texttt{2222} \\
UTS namespace & shared with the host & \textit{modified using \texttt{/usr/bin/unshare --uts}} \\
kernel version & Linux 3.16.0-4-amd64 & Linux 2.6.56-4-amd64 \\
umask & 0022 & 0002 \\
CPU type & \multicolumn{2}{l}{same for both builds \textit{(work in progress)}} \\
filesystem & \multicolumn{2}{l}{same for both builds \textit{(work in progress - disorderfs)}} \\
year, month, date & \multicolumn{2}{l}{same for both builds \textit{(work in progress)}} \\
hour, minute & \multicolumn{2}{l}{hour is usually the same… usually, the minute differs… \textit{(work in progress)}} \\
\textit{everything else} & \multicolumn{2}{l}{\textit{is likely the same…}}
    \end{tabular}
   }
  \end{table}
 \end{center}
\end{frame}



\begin{frame}
 \frametitle{Debian .buildinfo}

 \begin{itemize}
  \item Aggregate in the same file:
   \begin{itemize}
    \item Sources (checksums)
    \item Generated binaries (checksums)
    \item Packages used to build (with specific version, checksums coming soon)
   \end{itemize}
  \item Can be later used to reinstall environment exactly as it was
  \item For Debian all versions are available from \url{snapshot.debian.org}
 \end{itemize}
\end{frame}


\begin{frame}[fragile]
 \frametitle{Example .buildinfo}

{\small
\begin{verbatim}
Format: 1.9
Build-Architecture: amd64
Source: txtorcon
Binary: python-txtorcon
Architecture: all
Version: 0.11.0-1
Build-Path: /buildd/debian/txtorcon-0.11.0-1
Checksums-Sha256:
 a26549d9…7b 125910 python-txtorcon_0.11.0-1_all.deb
 28f6bcbe…69 2039 txtorcon_0.11.0-1.dsc
Build-Environment:
 base-files (= 8),
 base-passwd (= 3.5.37),
 bash (= 4.3-11+b1),
 …
\end{verbatim}
}
\end{frame}



{
\usebackgroundtemplate{%
 \begin{tikzpicture}[remember picture,overlay]%
  \node[shift={(-0.15\paperwidth, 0.4\paperheight)},at=(current page.south east)] {
    \includegraphics[width=0.2\paperwidth]{images/diffoscope_logo.png}
  };
 \end{tikzpicture}%
}
\begin{frame}{diffoscope}
 \frametitle{Debugging problems: diffoscope}

 \begin{itemize}
  \item Examines differences \textbf{in depth}
  \item Outputs HTML or plain text showing the differences
  \item Recursively unpacks archives
  \item Seeks human readability:
   \begin{itemize}
    \item uncompresses PDF
    \item disassembles binaries
    \item unpacks Gettext files
    \item … \textit{easy to extend to new file formats}
   \end{itemize}
  \item Falls back to binary comparison
  \item Available in Debian sid and stretch
  \item Maintainers in other distros wanted
 \end{itemize}
 \vfill
 \begin{center}
  \url{http://diffoscope.org/}\\
  {\footnotesize \color{gray}{(formely known as \texttt{debbindiff})}}
 \end{center}
\end{frame}
}

\begin{frame}
 \frametitle{diffoscope example (HTML output)}

 \begin{center}
  \includegraphics[width=0.9\paperwidth]{images/diffoscope_example_html.png}
 \end{center}
\end{frame}

\begin{frame}
 \frametitle{diffoscope example (text output)}

 \begin{center}
  \includegraphics[width=0.9\paperwidth]{images/diffoscope_example_text.png}
 \end{center}
\end{frame}

\begin{frame}
 \frametitle{\texttt{diffoscope} is "just" for debugging}

 \begin{itemize}
  \item Reminder: \texttt{diffoscope} is for \textbf{debugging}
  \item\only<2>{ "reproducible" according to our definition means: \textbf{bit by bit
  identical}. So the tools for testing whether something is reproducible are
  either \texttt{diff} or \texttt{sha256sum}!}
 \end{itemize}
\end{frame}


\begin{frame}
 \frametitle{\texttt{SOURCE\_DATE\_EPOCH}}

 \begin{itemize}
  \item Build date usually not useful for the user
  \item Standardize a build-time environment variable
  \item Value of \texttt{SOURCE\_DATE\_EPOCH} is used instead of the current date
  \item General solution for other free software projects and distributions
  \item In Debian, set from the latest \texttt{debian/changelog} entry
 \end{itemize}
\end{frame}

\begin{frame}
 \frametitle{\texttt{SOURCE\_DATE\_EPOCH} (closed bugs)}

 \begin{itemize}
  \item \sout{\texttt{\#791823}}: debhelper
  \item \sout{\texttt{\#787444}}: help2man
  \item \sout{\texttt{\#790899}}: epydoc
  \item \sout{\texttt{\#794004}}: ghostscript
  \item \sout{\texttt{\#783475}}: texi2html
  \item \sout{\texttt{\#794586}}: ocamldoc
  \item sphinx \small{\url{https://github.com/sphinx-doc/sphinx/pull/1954}}
 \end{itemize}

\end{frame}

\begin{frame}
 \frametitle{\texttt{SOURCE\_DATE\_EPOCH} (open bugs)}

 \begin{itemize}
  \item gcc (\texttt{\_\_DATE\_\_} and \texttt{\_\_TIME\_\_} macros) \texttt{\footnotesize{\url{https://gcc.gnu.org/ml/gcc-patches/2015-06/msg02210.html}}}
  \item \texttt{\#792687}: gettext (xgettext)
  \item \texttt{\#792201}: doxygen
  \item \texttt{\#800797}: docbook-utils
  \item \texttt{\#790801}: txt2man
  \item \texttt{\#791815}: libxslt
  \item \texttt{\#794681}: qt4-x11 (qthelpgenerator)
  \item \texttt{\#792202}: texlive-bin
 \end{itemize}

\end{frame}

\begin{frame}
 \frametitle{Work on individual packages}

 \begin{itemize}
  \item 644 bugs for individual problems
  \item 610 tagged with “patch”
  \item Nearly two patches per day since fall 2014 on average!
 \end{itemize}
\end{frame}

\begin{frame}[plain]
 \begin{tikzpicture}[remember picture,overlay]
  \node[at=(current page.center)] {
    \includegraphics[width=\paperwidth]{images/stats_bugs.png}
  };
 \end{tikzpicture}
\end{frame}

\section{Next?}

\begin{frame}
 \frametitle{Status and next steps in Debian}
 \begin{itemize}
  \item Remember: this is just a proof-of-concept, Debian is not 80\%
  reproducible.
  \item Major changes still need to be merged.
  \item Once this has happend, Debian will be >80\% reproducible.
  \item The next Debian release ("stretch") shall be >80\% reproducible.
 \end{itemize}
\end{frame}

\begin{frame}
 \frametitle{dpkg}

 \begin{itemize}\small
  \item \sout{\texttt{\#719844}: make compression of \{data,control\}.tar.gz deterministic}
  \item \texttt{\#759999}: set reproducible timestamps in \texttt{.deb} ar file headers
  \item \texttt{\#787980}: normalize file permissions when creating control.tar
  \item \texttt{\#719845}: make file order within {data,control}.tar.gz deterministic
  \item \texttt{dpkg-genbuldinfo}: \textit{patch already written, but waiting on agreement about spec}
 \end{itemize}
\end{frame}

\begin{frame}
 \frametitle{debhelper}

 \begin{itemize}\small
  \item \texttt{\#759886}: make mtimes of packaged files deterministic
  \item \sout{\texttt{\#759895}: add a call to
  \texttt{dh\_strip\_nondeterminism} in \texttt{dh}}
  \item \sout{\texttt{\#791823}: set \texttt{SOURCE\_DATE\_EPOCH} env var for
  reproducible builds}
 \end{itemize}
\end{frame}

\begin{frame}
 \frametitle{sbuild}

 \begin{itemize}\small
  \item \sout{\texttt{\#790868}: allow sbuild to use a deterministic build
  path to build packages}
  \item \texttt{\#778571}: predictible build location for reproducible builds
  \item Finish the \texttt{srebuild} script
 \end{itemize}
\end{frame}

\begin{frame}
 \frametitle{ftp.debian.org}

 \begin{itemize}\small
  \item \texttt{\#763822}: please include .buildinfo file in the archive
 \end{itemize}
\end{frame}

\begin{frame}
 \frametitle{"Finally", changing Debian policy}

 \begin{itemize}
  \item Section 4.15: “Sources \textbf{must} build reproducible binaries.” 
  \item\only<2-3> {I hope this will happen in early 2017 = after the Stretch
  (Debian 9) release}
  \item\only<3> {in 2016, hopefully: “Sources \textbf{shall} build reproducible binaries.”}
 \end{itemize}
\end{frame}

\begin{frame}
 \frametitle{Reproducible builds demand a defined build environment}
 \begin{itemize}
  \item Re-creating an identical build environment is mandatory too.
  \item Without an identical build environment reproducible builds
  will only happen by sheer luck, thus that's not really reproducible at all.
  \item\only<2>{This is probably only solved for Debian right now - and
  currently that's still a proof of concept only…}
 \end{itemize}
\end{frame}

\begin{frame}
 \frametitle{Reproducible builds are just the first 33\%}
 \begin{itemize}
  \item Continuous rebuilds need to happen in a systematic way and the
  resulting checksums need to be properly published.
  \item \only<2>{ Integration in end user tools\\
  "Do you really want to install this unreproducible software (y/N)" \\
  "Which rebuilders do you want to trust?"}
 \end{itemize}
\end{frame}

\begin{frame}
 \frametitle{Status of reproducible builds in other distros}

 \begin{itemize}
  \item As shown we're also testing Coreboot, OpenWrt, NetBSD, FreeBSD,
  Archlinux and soon Fedora.
  \item But the work needs to be done within those projects.
  \item \only<2>{And we are only testing for reproducible builds. No work
  has been done on the other
  66\% yet. (Systematic rebuilds and sharing the checksum \& end-user tool
  integration)}
 \end{itemize}
\end{frame}


\begin{frame}
 \frametitle{Android, Windows and all the rest}
 \begin{itemize}
  \item Android: some work on making Cyanogenmod reproducible (but with
  \texttt{faketime})
  \item Windows: why not, Microsoft is sharing\^ wselling the source at least… 
  \item IOS: in very short: it's a mess, might work with jailbreaked devices
  only - but AFAIK there's no source code anyway :-(
  \item MacOS: no idea, probably same as IOS :-(
  \item ...
 \end{itemize}
 \only<2> {Let's focus on Free Software first!}
\end{frame}


\section{Want to help?}

\begin{frame}
 \frametitle{As a software developer}
 \begin{itemize}
  \item stop using build date
  \item use \texttt{SOURCE\_DATE\_EPOCH} instead \\
  \item see  \url{https://reproducible-builds.org/specs/}
 \end{itemize}
\end{frame}

\begin{frame}
 \frametitle{Get involved - learning by doing}

 \begin{itemize}
  \item Test for yourself:
   \begin{itemize}
    \item just build something twice, run diffoscope on the results
    \begin{itemize}
     \item for better results use our “reproducible” repository, \texttt{pbuilder} and a custom config
    \end{itemize}
   \end{itemize}

  \item Tips on the wiki: \\
    {\small \url{https://wiki.debian.org/ReproducibleBuilds/Howto}} \\
    {\small
    \url{https://wiki.debian.org/ReproducibleBuilds/ExperimentalToolchain}}
  \item Ask for help on \texttt{\#debian-reproducible} \\
   or on the mailing-list
 \end{itemize}
\end{frame}

\begin{frame}
 \frametitle{Join the team!}

 \begin{itemize}
  \item Why?
   \begin{itemize}
    \item \heartsuit{}\heartsuit{}\heartsuit{} Lovely group of people \heartsuit{}\heartsuit{}\heartsuit{}
    \item Learn something new everyday
    \item Change the (software) world!
   \end{itemize}
  \item What do we do?
   \begin{itemize}
    \item Review packages
    \item Identify issues and document solutions
    \item \texttt{reproducible.d.n}, diffoscope, strip-nondeterminism
    \item Propose changes for toolchain
    \item Submit patches for individual packages
    \item Write more general documentation and  talk to the world
   \end{itemize}
 \end{itemize}
\end{frame}

\begin{frame}
 \frametitle{Create a new team!}

 \begin{itemize}
  \item Why?
   \begin{itemize}
    \item Every distribution should be reproducible!
    \item Learn something new everyday
    \item Change the (software) world!
   \end{itemize}
  \item How to get started?
   \begin{itemize}
    \item Talk to me here or talk to us on IRC or via mail.
    \item RTFM, there is lots of documentation
    \item Experiment - learning by doing
   \end{itemize}
 \end{itemize}
\end{frame}

\begin{frame}
 \frametitle{Write a thesis!}

 \begin{itemize}
  \item We are trying to document all our work, but we won't write
  scientific articles.
  \item Maybe you can do that?
  \item We'll be happy to help and there is Gunnar Wolf (gwolf@debian.org) in
  Mexico, D.F. too!
 \end{itemize}
\end{frame}


\section{Questions?}

\begin{frame}
 \frametitle{Questions, comments, ideas?}
 \begin{center}
 \end{center}
 \begin{itemize}
 \item\url{https://reproducible.debian.net}
 \item\url{#debian-reproducible} on \url{irc.OFTC.net}
 \end{itemize}
\end{frame}

\begin{frame}
 \frametitle{Thanks!}

 \begin{itemize}
  \item Debian “Reproducible Builds” team \\
        {\small (you are just \textbf{so} awesome!)}
  \item Linux Foundation and the Core Infrastructure Initiative
  \item Mini DebConf Cambridge 2015
\end{itemize}

 \begin{center}
  \includegraphics[height=0.1\paperheight]{images/linux_foundation_logo.png}
  \hspace{0.1\paperwidth}
  \includegraphics[height=0.1\paperheight]{images/cii_logo.png}
 \end{center}

 \vfill
 \begin{center}
  \resizebox{0.8\textwidth}{!}{%
   \begin{tabular}{rl}
    \texttt{holger@debian.org} & \texttt{B8BF 5413 7B09 D35C F026} \\
                               & \texttt{FE9D 091A B856 069A AA1C}
    \texttt{lamby@debian.org} & \texttt{C2FE 4BD2 71C1 39B8 6C53} \\
                              & \texttt{3E46 1E95 3E27 D431 1E58} \\
   \end{tabular}
  }
 \end{center}
\end{frame}

\end{document}
