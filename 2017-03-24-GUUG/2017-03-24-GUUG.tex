\documentclass[14pt]{beamer}
\setbeamertemplate{caption}[numbered]
\setbeamertemplate{caption label separator}{:}
\setbeamercolor{caption name}{fg=normal text.fg}
\usepackage{amssymb,amsmath}
\usepackage{ifxetex,ifluatex}
\usepackage{fixltx2e} % provides \textsubscript
\usepackage{lmodern}
\ifxetex
  \usepackage{fontspec,xltxtra,xunicode}
  \defaultfontfeatures{Mapping=tex-text,Scale=MatchLowercase}
  \newcommand{\euro}{€}
\else
  \ifluatex
    \usepackage{fontspec}
    \defaultfontfeatures{Mapping=tex-text,Scale=MatchLowercase}
    \newcommand{\euro}{€}
  \else
    \usepackage[T1]{fontenc}
    \usepackage[utf8]{inputenc}
      \fi
\fi
% use upquote if available, for straight quotes in verbatim environments
\IfFileExists{upquote.sty}{\usepackage{upquote}}{}
% use microtype if available
\IfFileExists{microtype.sty}{\usepackage{microtype}}{}
\PassOptionsToPackage{hyphens}{url}
\usepackage{hyperref}
\usepackage{ulem}

% Comment these out if you don't want a slide with just the
% part/section/subsection/subsubsection title:
\AtBeginPart{
  \let\insertpartnumber\relax
  \let\partname\relax
  \frame{\partpage}
}
\AtBeginSection{
  \let\insertsectionnumber\relax
  \let\sectionname\relax
  \begin{frame}[plain]
    \tableofcontents[currentsection]
  \end{frame}
}
\AtBeginSubsection{
  \let\insertsubsectionnumber\relax
  \let\subsectionname\relax
  \frame{\subsectionpage}
}

\setlength{\parindent}{0pt}
\setlength{\parskip}{6pt plus 2pt minus 1pt}
\setlength{\emergencystretch}{3em}  % prevent overfull lines
\setcounter{secnumdepth}{0}
% Thanks Richard Darst on how to get a nice Beamer theme.
% See http://rkd.zgib.net/wiki/DebianBeamerThemes

\usepackage{multicol}
\usepackage[absolute,overlay]{textpos}
\usepackage{tikz}
\usepackage{ctable}
\usetikzlibrary{positioning}

\usebackgroundtemplate{\includegraphics[width=\paperwidth]{images/swirl-lightest.pdf}}
\newif\ifplacelogo
\placelogotrue
\logo{\ifplacelogo\includegraphics[viewport=274 335 360 440,width=1cm]{images/openlogo-nd.pdf}\fi}

\definecolor{debianred}{rgb}{.780,.000,.211} % 199,0,54
\definecolor{debianblue}{rgb}{0,.208,.780} % 0,53,199
\definecolor{debianlightbackgroundblue}{rgb}{.941,.941,.957} % 240,240,244
\definecolor{debianbackgroundblue}{rgb}{.776,.784,.878} % 198,200,224

\usetheme{Boadilla}
\setbeamertemplate{navigation symbols}{}

\usecolortheme[named=debianbackgroundblue]{structure}
\setbeamercolor{normal text}{fg=black}
\setbeamercolor{titlelike}{fg=debianblue}
\setbeamercolor{sidebar}{fg=debianred,bg=debianbackgroundblue}

\setbeamercolor{palette sidebar primary}{fg=debianred}
\setbeamercolor{palette sidebar secondary}{fg=debianred}
\setbeamercolor{palette sidebar tertiary}{fg=debianred}
\setbeamercolor{palette sidebar quaternary}{fg=debianred}

\setbeamercolor{section in toc}{fg=debianred}
\setbeamercolor{subsection in toc}{parent=debianred}

\setbeamercolor{item}{fg=debianred}

\setbeamercolor{block title}{fg=debianblue}


\title[Reproducible Builds everywhere]{Reproducible
builds everywhere}
\subtitle{Bit by bit identical binaries \\
from a given source}
\author[h01ger]{%
   \texorpdfstring{
            \centering
            Holger 'h01ger' Levsen
   }{h01ger}}
\date[GUUG 2017]{%
 GUUG Frühjahrsfachgespräch 2017\\
Darmstadt, Germany\\
 \small{2017-03-24}}

\begin{document}
\placelogofalse

\begin{frame}[plain]
 \titlepage
\end{frame}

\placelogotrue

\begin{frame}
 \frametitle{about h01ger}

 \begin{itemize}
  \item \small{\texttt{B8BF 5413 7B09 D35C F026  FE9D 091A B856 069A AA1C}}
  \item Debian user since 1995, contributor since 2001, official developer
  status since 2007
  \item DebConf organizer,
  founded the DebConf video team
   \begin{itemize}
    \item \texttt{http://video.debian.net}
   \end{itemize}
 \item Debian-Edu (Debian for education)
  \item Debian QA (quality assurance)
  \begin{itemize}
   \item \texttt{https://piuparts.debian.org}
   \item \texttt{https://jenkins.debian.net} (~1400 jobs continously testing Debian)
  \end{itemize}
  \item Debian Reproducible builds team member
  \begin{itemize}
   \item since April 2015 funded by the Linux Foundation
 \end{itemize}
  \item<2> Ich spreche auch Deutsch… sometimes.
 \end{itemize}
\end{frame}

\begin{frame}
 \frametitle{Debian reproducible builds contributors}
 \begin{center}
  \begin{columns}
   \footnotesize
   \column{.30\linewidth}
    {akira} \\
    {Alexis Bienvenüe} \\
    {Andrew Ayer} \\
    {Asheesh Laroia} \\
    Boyuan Yang \\
    {Ceridwen} \\
    {Chris Lamb} \\
    {Chris West} \\
    {Christoph Berg} \\
    Clint Adams \\
    Dafydd Harries \\
    {Daniel Kahn Gillmor} \\
    {Daniel Shahaf} \\
    Daniel Stender \\
    David Suarez \\
    {Dhole} \\
    Drew Fisher \\
    Emmanuel Bourg \\
    \column{.30\linewidth}
    Emanuel Bronshtein \\
    Esa Peuha \\
    {Fabian Wolff} \\
    {Guillem Jover} \\
    Hans-Christoph Steiner \\
    Harlan Lieberman-Berg \\
    {Helmut Grohne} \\
    \only<1>{Holger Levsen}\only<2>{{\color{debianred} Holger Levsen}} \\
    HW42 \\
    Intrigeri \\
    {Jelmer Vernooij} \\
    {josch} \\
    Juan Picca \\
    {Lunar} \\
    Maria Glukhova \\
    Mathieu Bridon \\
    {Mattia Rizzolo} \\
    Nicolas Boulenguez \\
    {Niels Thykier} \\
   \column{.30\linewidth}
    Niko Tyni \\
    {Paul Wise} \\
    Peter De Wachter \\
    Philip Rinn \\
    {Reiner Herrmann} \\
    Robbie Harwood \\
    {Santiago Vila} \\
    {Sascha Steinbiss} \\
    {Satyam Zode} \\
    {Scarlett Clark} \\
    {Stefano Rivera} \\
    {Stéphane Glondu} \\
    {Steven Chamberlain} \\
    Tom Fitzhenry \\
    {Valerie Young} \\
    Valentin Lorentz \\
    {Wookey} \\
    {Ximin Luo} \\
  \end{columns}
 \end{center}
\end{frame}


\placelogofalse

\begin{frame}
 \frametitle{Who are you?}
 \begin{itemize}
  \item<2-6> Seen a talk about reproducible builds?
  \item<3-6> Contributed to the effort?
  \item<4-6> Uses Debian or a Debian based systems?
  \item<5-6> Uses Fedora, RHEL, CentOS or a Fedora derivative based systems?
  \item<6> BSD?
 \end{itemize}
\end{frame}



\section{Motivation}

\begin{frame}[fragile]
 \frametitle{The problem: we need to believe}
 \begin{itemize}
  \item Free Software is great: one can study, modify, share and use it!
  \item<2-4> We study, modify and share source code.
  \item<2-4> We use binaries.
  \item<3-4> We need to believe our binaries come from the source code they are said to made from.
  \item<4> \textbf{I do not want to believe.}
 
 \end{itemize}
\end{frame}

\begin{frame}
 \frametitle{The problem in greater detail}

 \begin{center}
  \includegraphics[width=0.7\textwidth]{images/31c3.png}

  Available on \url{media.ccc.de}, 31c3
 \end{center}
\end{frame}

\begin{frame}[fragile]
 \frametitle{A few examples from that 31c3 talk}
 \begin{itemize}
  \item CVE-2002-0083: remote root exploit in \texttt{sshd}, a single bit difference in the binary
  \item<2-5> 31c3 talk had a live demo with a kernel module modifying source code in memory only
  \item<3-5> How can you be sure what's running on your machine or on a build
  daemon network connected to the net? Do you ever leave your computers
  physically alone? 
  \item<4-5> How much do you pay your admins? Enough to withstand a multi million
  dollar attack?
  \item<5> Legal challenges. Could you be forced to backdoor (some of) your
  software (for some customers)?
 \end{itemize}
\end{frame}

\begin{frame}[fragile]
 \frametitle{Another example from real life}

 At a CIA conference in 2012:
 \begin{center}
  \includegraphics[width=0.8\textwidth]{images/strawhorse.png}

  {\footnotesize
  \url{firstlook.org/theintercept/2015/03/10/ispy-cia-campaign-steal-apples-secrets/}
  }
 \end{center}
\end{frame}


\begin{frame}
 \frametitle{The solution}

 \begin{center}
 \Large{
 Promise that anyone can always and independently generate
 identical binary packages from a given source}
\end{center}
\end{frame}


\begin{frame}
 \frametitle{The solution}

 \begin{center}
 We call this:

 \Huge{ “Reproducible builds” }
 \end{center}
\end{frame}

\placelogotrue

\begin{frame}
 \frametitle{Debian demo}
 \begin{itemize}
 \item Build a package 5 times, get 5 .debs with different checksums
 \item<2-3> Build a package 5 times, get 5 .debs with the same checksum\\
 \item<3>{Yes, it's really this simple.}
 \end{itemize}
% show this once running in plain sid,
% and then in sid with our modified toolchain.
%
% prepare demo:
% mkdir demo ; cd demo ; apt-get source giftrans
%
% do demo:
% PTH=$(mktemp -d); OPTH=$PWD; P=giftrans; cp ${P}_* $PTH/; cd $PTH ;
%   dpkg-source -x ${P}*.dsc ; for X in 1 2 3 4 5 ; do (cd ${P}-*/;
%   dpkg-buildpackage -b -uc -us); mkdir -p .$X ; cp $P_*.deb .$X; done ; rm
%   *.deb ; echo; sha1sum *dsc *z .*/*.deb | grep -v giftrans-dbgsym ; cd - ;
% rm -r $PTH
\end{frame}

\placelogofalse

\begin{frame}[plain]
\begin{center}
 \Huge{This should become the \textbf{norm}.}

 \visible<2>{\small{ We want to change the meaning of "free software":

  it's only free software if it's reproducible!}}
\end{center}
\end{frame}

\begin{frame}[fragile]
 \frametitle{More benefits than "just" security…}
 \begin{itemize}
  \item Lots and lots of QA benefits - we've found so many subtile bugs.
  \item<2-5> Google does reproducible builds, to save time and money.
  \item<3-5> Smaller deltas, thus faster updates possible (for packages and
  images).
  \item<4-5> Side effect: meaningful binary diff between two versions.
  \item<5> …
 \end{itemize}
\end{frame}


\section{Common ressources}

\begin{frame}
 \frametitle{reproducible-builds.org}

 \begin{itemize}
  \item \texttt{https://reproducible-builds.org}
  \item git repositories, IRC channels, mailinglists, webspace
 \end{itemize}
 \begin{center}
 \includegraphics[width=0.7\textwidth]{images/rbwww1.png}
 \end{center}
\end{frame}


{
\usebackgroundtemplate{%
 \begin{tikzpicture}[remember picture,overlay]%
  \node[shift={(-0.1\paperwidth, 0.15\paperheight)},at=(current page.south east)] {
    \includegraphics[width=0.2\paperwidth]{images/diffoscope_logo.png}
  };
 \end{tikzpicture}%
}

\begin{frame}{diffoscope}
 \frametitle{Debugging problems: \texttt{https://try.diffoscope.org}}

 \begin{itemize}
  \item Examines differences \textbf{in depth}.
  \item Recursively unpacks archives, uncompresses PDFs, disassembles
  binaries, unpacks Gettext files, …
  \item Easy to extend to new file formats.
  \item Falls back to binary comparison.
  \item Outputs HTML or plain text with human readable differences.
  \item Available from \texttt{git}, PyPI, Debian, \\
   Arch Linux, Guix, Homebrew, Fedora. Works on BSD.
  \item Maintainers in other distros wanted.
  \item \url{https://diffoscope.org/}
 \end{itemize}
\end{frame}


\begin{frame}
 \frametitle{\texttt{diffoscope} example (HTML output)}
 \begin{tikzpicture}[remember picture]
  \node[at=(current page.center)] {
   \includegraphics[width=0.9\paperwidth]{images/diffoscope_example_html.png}
  };
 \end{tikzpicture}
\end{frame}


\begin{frame}
 \frametitle{\texttt{diffoscope} is "just" for debugging}

 \begin{itemize}
  \item Reminder: \texttt{diffoscope} is for \textbf{debugging}
  \item "reproducible" according to our definition means: \textbf{bit by bit
  identical}. So the tools for testing whether something is reproducible are
  either \texttt{diff} or \texttt{sha256sum}!
  \item<2> \texttt{https://try.diffoscope.org}
 \end{itemize}
\end{frame}

}



\placelogotrue


\begin{frame}
 \frametitle{tests.reproducible-builds.org}

 \begin{itemize}
  \item Continuously testing Debian \texttt{testing}, \texttt{unstable} and
  \texttt{experimental}
  \item Also testing: coreboot, OpenWrt, LEDE, NetBSD, FreeBSD,
  Arch Linux, Fedora and soon F-Droid too
  \item 46 nodes (amd64/i386/arm64/armhf), >200 cores and >1 TB RAM
  \item 502 jenkins jobs running on jenkins.debian.net
  \item 43 scripts in Python and Bash, 283 lines of code in average
  \item 37 contributors for \texttt{jenkins.debian.net.git}
 \end{itemize}
\end{frame}


\begin{frame}[fragile]
 \frametitle{Variations (when testing Debian)}

 \begin{center}
  \begin{table}
   \resizebox{0.95\textwidth}{!}{%
    \begin{tabular}{l|ll}
\textbf{variation} & \textbf{first build} & \textbf{second build} \\
\hline
hostname & \texttt{jenkins} & \texttt{i-capture-the-hostname} \\
domainname & \texttt{debian.net} & \texttt{i-capture-the-domainname} \\
\texttt{env TZ} & \texttt{GMT+12} & \texttt{GMT-14} \\
\texttt{env LANG} & \texttt{C} & \texttt{fr\_CH.UTF-8} \\
\texttt{env LC\_ALL} & not set & \texttt{fr\_CH.UTF-8} \\
\texttt{env USER} & \texttt{pbuilder1} & \texttt{pbuilder2} \\
uid & \texttt{1111} & \texttt{2222} \\
gid & \texttt{1111} & \texttt{2222} \\
UTS namespace & shared with the host & \textit{modified using \texttt{/usr/bin/unshare --uts}} \\
kernel version & Linux 3.16 or 4.X & on amd64 and arm64 always varied \\
 & & on armhf sometimes \\
 & & on i386 32/64bit kernel variation instead \\
umask & 0022 & 0002 \\
CPU type & & varied on i386: Intel or AMD CPU \\
 & & on armhf varied a bit \\
 & & not varied on amd64 nor arm64 \\
filesystem & \texttt{tmpfs} & same for both builds on amd64, i386 and arm64 \\
 & & on armhf \texttt{ext3/4} \\
 & & \textit{(and we have} \texttt{disorderfs}\textit{, but the code is disabled)} \\
year, month, date & \multicolumn{2}{l}{on amd64, arm64 and i386: 398 days variation, on armhf not yet} \\
hour, minute & \multicolumn{2}{l}{hour and minute deterministically and non-deterministically varied… } \\
\textit{everything else} & \multicolumn{2}{l}{\textit{is likely the same…}}
    \end{tabular}
   }
  \end{table}
 \end{center}
\end{frame}

\placelogofalse

\begin{frame}
 \frametitle{Common problems}

 \begin{itemize}
  \item time stamps
  \item timezones
  \item locales
  \item build paths
  \item everything else (seperated into known issues and the blurry rest)
 \end{itemize}
\end{frame}

\begin{frame}
 \frametitle{Documentation about common problems}
 \begin{itemize}
  \item \texttt{https://reproducible-builds.org/docs}
  \item Lunar's talk from CCCamp 2015 also on
  \texttt{https://media.ccc.de}
 \begin{tikzpicture}[remember picture]
  \node[shift={(-1.05\paperwidth, -0.3\paperheight)},at=(current page.south east)] {
    \includegraphics[width=0.83\textwidth]{images/cccamp2015_lunar_random.png}
  };
 \end{tikzpicture}
 \end{itemize}
\end{frame}


\begin{frame}
 \frametitle{\texttt{SOURCE\_DATE\_EPOCH}}

 \begin{itemize}
  \item Build date (timestamps) usually not useful for the user
  \item \texttt{SOURCE\_DATE\_EPOCH} is defined as the last modification of
  the source, since the epoch (1970-01-01)
  \item can be used instead of current date
  \item can also be used for random seeds etc.
  \item in Debian, set from the latest \texttt{debian/changelog} entry
  \item can be set based on the latest git commit or the latest file
  modification date too
 \end{itemize}
\end{frame}

\begin{frame}
 \frametitle{\texttt{SOURCE\_DATE\_EPOCH}}

 \begin{itemize}
  \item \texttt{SOURCE\_DATE\_EPOCH} spec available:
  \item \texttt{https://reproducible-builds.org/specs/}
  \item many upstreams support it already
  \item has been adopted by other distributions
  (openSUSE, OpenWrt, LEDE, NetBSD, FreeBSD, Arch Linux, coreboot, Guix, …) and many many
  upstreams (GCC, dpkg, rpm, mkisofs, ghostscript, libxslt, sphinx,
  texlive-bin, …)
 \end{itemize}
\end{frame}

\begin{frame}
 \frametitle{two more tools}

 \begin{itemize}
  \item \texttt{strip-nondeterminism} 
  \item<2> \texttt{reprotest} 
 \end{itemize}
\end{frame}

\placelogotrue

\section{Status Debian}

\begin{frame}
 \frametitle{Progress in Debian \texttt{testing} ("stretch")}
 \begin{tikzpicture}[remember picture]
  \node[shift={(-0.5\paperwidth, \paperheight)},at=(current page.south east)] {
    \includegraphics[height=0.65\paperheight]{images/stats_pkg_state_testing.png}
  };
 \end{tikzpicture}
 \begin{center}
  \footnotesize{23,378 (93.8\%) out of 24,909 source packages are reproducible \\
    in our test framework on \texttt{amd64}}
  \vfill
 \end{center}
\end{frame}

\begin{frame}
 \frametitle{Progress in Debian \texttt{unstable}}
 \begin{tikzpicture}[remember picture]
  \node[shift={(-0.5\paperwidth, \paperheight)},at=(current page.south east)] {
    \includegraphics[height=0.65\paperheight]{images/stats_pkg_state_unstable.png}
  };
 \end{tikzpicture}
 \begin{center}
  \footnotesize{20,597 (79.2\%) out of 25,982 source packages are reproducible \\
    in our test framework on \texttt{amd64}} (difference due to build path variations)
  \vfill
 \end{center}
\end{frame}

\begin{frame}
 \frametitle{\texttt{BUILD\_PATH\_PREFIX\_MAP}}

 \begin{itemize}
  \item Those 93.8\% in Stretch are nice, but…
  \item We want to be able to build in any path.
  \item 15-20\% of the packages embed build-time paths into generated files, evevn though these paths do not exist at runtime, nor do they exist in the source code.
  \item \texttt{BUILD\_PATH\_PREFIX\_MAP} spec available, though we have formally released it yet…
  \item \texttt{https://reproducible-builds.org/specs/}
  \item Example patches exist, though this is still work in progress.
 \end{itemize}
\end{frame}


\begin{frame}
 \frametitle{Details on tests.reproducible-builds.org}

 \begin{itemize}
  \item \url{https://reproducible.debian.net/$src}
  \item 48 package sets 
  \item 292 categorised distinct issues
  \item 6,604 notes
  \item 1,473 unreproducible packages in \texttt{stretch/amd64} (testing), but only
  90 without a note (5,253 in \texttt{unstable} but also only 149 without a
  note)
  \item maintained in \texttt{notes.git} by 49 contributors
  \item currently Debian only, but cross distro notes are planned
 \end{itemize}
\end{frame}


\begin{frame}
 \frametitle{Debian \texttt{.buildinfo} files}

 \begin{itemize}
  \item Aggregates in the same file:
   \begin{itemize}
    \item Sources (checksums)
    \item Generated binaries (checksums)
    \item Packages used to build (with specific version, checksums coming soon)
   \end{itemize}
  \item Can be later used to exactly recreate environment
  \item For Debian, all versions are available from \url{snapshot.debian.org}
 \end{itemize}
\end{frame}


\begin{frame}
 \frametitle{Progress in the Debian bug tracker}
 \begin{tikzpicture}[remember picture]
  \node[shift={(-0.5\paperwidth, \paperheight)},at=(current page.south east)] {
    \includegraphics[height=0.65\paperheight]{images/stats_bugs_sin_ftbfs_state.png}
  };
 \end{tikzpicture}
 \begin{center}
  \footnotesize{As a rule, we file bugs with patches. \\
  There are very few exceptions.}
  \vfill
 \end{center}
\end{frame}

\begin{frame}
 \frametitle{Sending progress upstream}
 \begin{itemize}
 \item So we filed a lot of bugs… with patches…! And 1763 were even closed.
 \item … but only in Debian and we rely on Debian maintainers sending them
 upstream.
 \item<2-3> Bernard Wiedemann (from openSUSE) thought that wasn't good enough
 and created \texttt{https://github.com/orgs/distropatches}
 \item<3> Once Debian 10, "buster" development starts, we plan to tackle those 547 open bugs ones…
 \end{itemize}
\end{frame}



\begin{frame}
 \frametitle{Debian summary / What's left to do}
 \begin{itemize}
  \item This is/was a proof-of-concept, Debian is neither 93.8\% reproducible nor
  79.2\%. (and 10\% > 2,500 sources packages!)
  \item<2-3> All our required changes are finally in Debian now!
  \item<2-3> Debian 9, "stretch", has 93\% reproducible sources, but only one third of the binary packages are…
  \item<2-3> Because, Debian does not (yet?) do full rebuilds before
  releasing… so stuff is in the archive which is not reproducible unless it's
  rebuild.
  \item<3> And then we don't distribute \texttt{.buildinfo} files yet.
   That (and user tools) still needs more \it{design} and code.
 \end{itemize}
\end{frame}


\begin{frame}
 \frametitle{Debian summary continued}
 \begin{itemize}
  \item Debian 9, "stretch", is mostly reproducible (from source).
  \item Canonical can take our work now and make Ubuntu 17.04
  (partyl) reproducible…
  \item<2-4> Security updates for "stretch" can+should be reproducible!
  \item<3-4> Debian 10, "buster", will be our first reproducible release, few exceptions expected.
  \item<4> We hope \texttt{debian-policy} will mandate 100\%
  reproducible builds for Debian 11, "bullseye", with development starting in 2019.
 \end{itemize}
\end{frame}

\begin{frame}
 \frametitle{Tell the world \& collaborate}

 \begin{itemize}
  \item "We don't care about Debian (only), we care about free and open
   source software."
  \item<2-3> 99 Weekly reports since May 2015
   \begin{itemize}
    \item started by Lunar
    \item nowadays published weekly by Chris and Ximin
    \item \url{https://reproducible.alioth.debian.org/blog/}
  \item<3> so next week will be the \textbf{100th} !!
   \end{itemize}
    \end{itemize}
\end{frame}




\begin{frame}
 \frametitle{Tell the world \& collaborate (continued)}

 \begin{itemize}
  \item First Reproducible World Summit in December 2015 (Athens, Greece)
   \begin{itemize}
    \item \texttt{reproducible.debian.net} became \texttt{tests.reproducible-builds.org}
   \end{itemize}
    \item Second Reproducible World Summit in December 2016 in Berlin
    \item Reproducible Builds Hamburg Hackathon 2017, \textbf{5-7th of May}
    \item Third summit in December 2017?
   \item<2> GSoC and Outreachy
 \end{itemize}
\end{frame}



\section{Status Non-Debian World}

\placelogofalse

\begin{frame}
 \frametitle{Skipping some…}
 \begin{itemize}
  \item \texttt{https://tests.r-b.org/coreboot}
  \item \texttt{https://tests.r-b.org/lede}
  \item \texttt{https://tests.r-b.org/openwrt}
  \item almost: \texttt{https://tests.r-b.org/f-droid}
  \item paused: \texttt{https://tests.r-b.org/archlinux}
 \end{itemize}
 \begin{center}
  \includegraphics[height=0.13\paperheight]{images/coreboot.png}
  \hspace{0.05\paperwidth}
  \includegraphics[height=0.15\paperheight]{images/lede.png}
  \hspace{0.05\paperwidth}
  \includegraphics[height=0.3\paperheight]{images/openwrt.png}
  \hspace{0.05\paperwidth}
  \includegraphics[height=0.13\paperheight]{images/f-droid.png}
  \hspace{0.05\paperwidth}
  \includegraphics[height=0.13\paperheight]{images/archlinux.png}
\end{center}
\end{frame}


\begin{frame}
 \frametitle{Skipping some more…}
 \begin{itemize}
\item Cygnus.com (1992)
\item Bitcoin (2011)
\item Tor (2013)
\item<2-3> NixOS, GNU Guix
\item<2-3> openBSD, ElectroBSD
\item<2-3> Qubes, Tails, webconverger
\item<2-3> Google Bazil, docker
\item<3> ducible (build tool for Windows)
\item<3> very few commercial, propietary software
\item<3> Signal
\item<3> Shim (secure-boot)
 \end{itemize}
\end{frame}


\begin{frame}
 \frametitle{Detour: what, reproducible commercial Software???}
 \begin{itemize}
\item Guess which
\item <2-3>   windows? (the source is available)
\item <2-3>   medical devices in your body?
\item <2-3>   arms?
\item <2-3>   critical infrastructure like in nuclear powerplants?
\item <2-3>   cars?
\item <3> Gambling machines!
 \end{itemize}
\end{frame}

\begin{frame}
 \frametitle{FreeBSD vs NetBSD}
 \begin{itemize}
  \item \texttt{https://tests.r-b.org/freebsd} at 99.6\%
  \item \texttt{https://tests.r-b.org/netbsd} reached 100\%
  \item<2> only base system built so far
  \item<2> NetBSD uses non-default settings to achieve this
  \item<2> ports planned
 \end{itemize}
 \begin{center}
  \includegraphics[height=0.13\paperheight]{images/freebsd.png}
  \hspace{0.05\paperwidth}
  \includegraphics[height=0.13\paperheight]{images/netbsd.png}
\end{center}
\end{frame}


\begin{frame}
 \frametitle{reproducible openSUSE}
 \begin{itemize}
  \item
  \small{\texttt{https://build.opensuse.org/package/show \\
  /home:bmwiedemann:reproducible/rpm?expand=0}}
  \item Bernhard Wiedemann has built openSUSE twice (with some variations):
  \begin{itemize}
   \item build-succeeded: 3172
   \item bit-by-bit-identical: 2117
   \item not-bit-by-bit-identical: 1055
 \end{itemize}
 \end{itemize}
 \begin{tikzpicture}[remember picture,overlay]
  \node[shift={(-0.1\paperwidth, 0.13\paperheight)},at=(current page.south east)] {
    \includegraphics[height=0.15\paperheight]{images/openSUSE.png}
  };
 \end{tikzpicture}
\end{frame}

\begin{frame}
 \frametitle{tests.r-b.org/fedora}
 \begin{itemize}
  \item used to test Fedora 23, could be made working again
  \item or build elsewhere and machine readable exported
  \end{itemize}
 \begin{tikzpicture}[remember picture,overlay]
  \node[shift={(-0.07\paperwidth, 0.13\paperheight)},at=(current page.south east)] {
    \includegraphics[height=0.15\paperheight]{images/fedora.png}
  };
 \end{tikzpicture}
\end{frame}

\begin{frame}
 \frametitle{Fedora basics}
 \begin{itemize}
  \item \texttt{diffoscope} is available in Fedora
  \item \texttt{yum} and \texttt{dnf} might create non-identical environments
  \item \texttt{rpm}-4.13 has an option to override hostname via rpmmacros
  \item signed RPMs -> re-apply signature, will match for identical builds
  \end{itemize}
 \begin{tikzpicture}[remember picture,overlay]
  \node[shift={(-0.07\paperwidth, 0.13\paperheight)},at=(current page.south east)] {
    \includegraphics[height=0.15\paperheight]{images/fedora.png}
  };
 \end{tikzpicture}
\end{frame}

\begin{frame}
 \frametitle{TODO: design \texttt{.buildinfo} files from koji/mock/zypper}
 \begin{itemize}
  \item rfc822 format?
  \item needs to define the environment
  \item needs to define the sources (input)
  \item needs to define the binaries (output)
 \end{itemize}
 \begin{tikzpicture}[remember picture,overlay]
  \node[shift={(-0.07\paperwidth, 0.13\paperheight)},at=(current page.south east)] {
    \includegraphics[height=0.15\paperheight]{images/fedora.png}
  };
 \end{tikzpicture}
\end{frame}



\section{Future work}

\begin{frame}
 \frametitle{Future work}
 \begin{itemize}
 \item<1-3> So far we mostly worked on making reproducible builds possible…
 \item<2-3> We'll need constant tests for future code.
 \item<3> And then, this still needs tools, infrastructure and policies to become
 meaningful and to be used in practice.
 \end{itemize}
\end{frame}

\begin{frame}
 \frametitle{Rebuilds and sharing signed checksums}
 \begin{itemize}
  \item Almost no work has been done here yet. We are just at the first step:
  being able to rebuild reproducibly…
  \item Different projects, different solutions?
 \begin{itemize}
  \item<2> something like \texttt{.buildinfo} files (defining the environment,
  the input and the output(s)) will be needed everywhere:
  \item<2> implemented for Debian (both in sbuild and well as
  buildinfo.debian.net)
  \item<2> work has begun for coreboot, LEDE/OpenWrt and Fedora (mock/koji)
  and maybe openSUSE (OpenBuildService)
 \end{itemize}
 \end{itemize}
\end{frame}

\begin{frame}
 \frametitle{Rebuilders and sharing signed checksums, cont.}
 \begin{itemize}
  \item Individuelly signed checksums (think web of trust) could work in the
  Debian case (we have a gpg web of trust), but IMO won't scale.
  \item { Another idea: rebuilders, run by large organisations,
  eg. ACLU, BSI, CCC, Deutsche Bank, Greenpeace, GUUG, NASA, NSA, etc…}
  \item Fedora rebuilds Debian, Debian rebuilds openSUSE, openSUSE rebuilds
  NetBSD, etc…
  \item Big customers could just rebuild everything themselves.
 \end{itemize}
\end{frame}


\begin{frame}
 \frametitle{Integration in user tools}
 \begin{itemize}
  \item "Do you really want to install this unreproducible software (y/N)"
  \item<2-3> "Do you want to build those packages which have unconfirmed checksums,
  before installing? (Y/n)"
  \item<3>{ "How many signed checksums do you require to call a package
  'reproducible'?" - and whom do you trust?}
 \end{itemize}
\end{frame}


\section{Getting involved}

\begin{frame}
 \frametitle{As a software developer}
 \begin{itemize}
  \item Stop using build dates
  \item Use \texttt{SOURCE\_DATE\_EPOCH} instead
  \item See \url{https://reproducible-builds.org/specs/}
 \end{itemize}
\end{frame}


\begin{frame}
 \frametitle{Form your reproducible builds team!}
 \begin{itemize}
  \item Why?
   \begin{itemize}
    \item Every distribution should be reproducible!
    \item Learn something new everyday
    \item Change the (software) world!
    \item \texttt{https://tests.reproducible-builds.org/\$distro} needs \textbf{your} help
   \end{itemize}
  \item How to get started?
   \begin{itemize}
    \item Build something twice, run \texttt{diffoscope} on the results.
    \item Experiment - learning by doing
    \item RTFM, there is lots of documentation
    \item Talk to me here or talk to us on IRC or via mail.
   \end{itemize}
 \end{itemize}
\end{frame}

\section{Feedback}

\placelogotrue

\begin{frame}
 \frametitle{Thank You!}

 \begin{itemize}
  \item
    {All “Reproducible Builds” contributors \\
        {\small (You are just \textbf{so} awesome!)}}

\end{itemize}

 \begin{center}
  \includegraphics[height=0.08\paperheight]{images/linux_foundation_logo.png}
  \hspace{0.05\paperwidth}
  \includegraphics[height=0.08\paperheight]{images/cii_logo.png}
  \hspace{0.05\paperwidth}
  \includegraphics[height=0.08\paperheight]{images/profitbricks_logo.png}
 \end{center}

 \vfill
 \begin{center}
  \resizebox{0.9\textwidth}{!}{%
   \begin{tabular}{rl}
    \texttt{holger@debian.org} & \texttt{B8BF 5413 7B09 D35C F026} \\
                               & \texttt{FE9D 091A B856 069A AA1C}
\end{tabular}
  }
 \end{center}
\end{frame}

\placelogofalse

\begin{frame}
 \frametitle{Questions, comments, ideas?}

 \begin{itemize}
  \item \url{https://reproducible-builds.org/}
  \item \texttt{\#reproducible-builds} on \texttt{irc.OFTC.net}
  \item \url{https://lists.reproducible-builds.org}
  \item twitter: @ReproBuild
  \item<2> Mike and Seth's talk from 31c3 about motivations
  \item<2> Lunar's talk about fixing reproducible issues from CCCamp 15
  \end{itemize}
\end{frame}

\placelogotrue

\begin{frame}{}
\begin{textblock}{12}(2, 6)
    \tiny{
      Copyright \copyright{} 2014--2017 \\
         Holger Levsen \texttt{holger@layer-acht.org} and others.\\[3.0mm]
      Copyright of images included in this document are held by
      their respective owners.
      \\[3.0mm]
      This work is licensed under the \alert{Creative Commons
        Attribution-Share Alike 3.0} License.  To view a copy of this
      license, visit
      \url{http://creativecommons.org/licenses/by-sa/3.0/} or send a
      letter to Creative Commons, 171 Second Street, Suite 300, San
      Francisco, California, 94105, USA.
      \\[2.0mm]
      % Give a link to the 'Transparent Copy', as per Section 3 of the GFDL.
      The source of this document is available from
      \url{https://anonscm.debian.org/git/reproducible/presentations.git}.
    }
  \end{textblock}
\end{frame}

\end{document}
