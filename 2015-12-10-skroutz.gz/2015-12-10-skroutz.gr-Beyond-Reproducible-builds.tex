\documentclass[14pt,aspectratio=169]{beamer}
\setbeamertemplate{caption}[numbered]
\setbeamertemplate{caption label separator}{:}
\setbeamercolor{caption name}{fg=normal text.fg}
\usepackage{amssymb,amsmath}
\usepackage{ifxetex,ifluatex}
\usepackage{fixltx2e} % provides \textsubscript
\usepackage{lmodern}
\ifxetex
  \usepackage{fontspec,xltxtra,xunicode}
  \defaultfontfeatures{Mapping=tex-text,Scale=MatchLowercase}
  \newcommand{\euro}{€}
\else
  \ifluatex
    \usepackage{fontspec}
    \defaultfontfeatures{Mapping=tex-text,Scale=MatchLowercase}
    \newcommand{\euro}{€}
  \else
    \usepackage[T1]{fontenc}
    \usepackage[utf8]{inputenc}
      \fi
\fi
% use upquote if available, for straight quotes in verbatim environments
\IfFileExists{upquote.sty}{\usepackage{upquote}}{}
% use microtype if available
\IfFileExists{microtype.sty}{\usepackage{microtype}}{}
\PassOptionsToPackage{hyphens}{url}
\usepackage{hyperref}
\usepackage{ulem}

% Comment these out if you don't want a slide with just the
% part/section/subsection/subsubsection title:
\AtBeginPart{
  \let\insertpartnumber\relax
  \let\partname\relax
  \frame{\partpage}
}
\AtBeginSection{
  \let\insertsectionnumber\relax
  \let\sectionname\relax
  \begin{frame}[plain]
    \tableofcontents[currentsection]
  \end{frame}
}
\AtBeginSubsection{
  \let\insertsubsectionnumber\relax
  \let\subsectionname\relax
  \frame{\subsectionpage}
}

\setlength{\parindent}{0pt}
\setlength{\parskip}{6pt plus 2pt minus 1pt}
\setlength{\emergencystretch}{3em}  % prevent overfull lines
\setcounter{secnumdepth}{0}
% Thanks Richard Darst on how to get a nice Beamer theme.
% See http://rkd.zgib.net/wiki/DebianBeamerThemes

\usepackage{multicol}
\usepackage{tikz}
\usepackage{ctable}
\usetikzlibrary{positioning}

\usebackgroundtemplate{\includegraphics[width=\paperwidth]{images/swirl-lightest.pdf}}
\logo{\includegraphics[viewport=274 335 360 440,width=1cm]{images/openlogo-nd.pdf}}

\definecolor{debianred}{rgb}{.780,.000,.211} % 199,0,54
\definecolor{debianblue}{rgb}{0,.208,.780} % 0,53,199
\definecolor{debianlightbackgroundblue}{rgb}{.941,.941,.957} % 240,240,244
\definecolor{debianbackgroundblue}{rgb}{.776,.784,.878} % 198,200,224

\usetheme{Boadilla}
\setbeamertemplate{navigation symbols}{}

\usecolortheme[named=debianbackgroundblue]{structure}
\setbeamercolor{normal text}{fg=black}
\setbeamercolor{titlelike}{fg=debianblue}
\setbeamercolor{sidebar}{fg=debianred,bg=debianbackgroundblue}

\setbeamercolor{palette sidebar primary}{fg=debianred}
\setbeamercolor{palette sidebar secondary}{fg=debianred}
\setbeamercolor{palette sidebar tertiary}{fg=debianred}
\setbeamercolor{palette sidebar quaternary}{fg=debianred}

\setbeamercolor{section in toc}{fg=debianred}
\setbeamercolor{subsection in toc}{parent=debianred}

\setbeamercolor{item}{fg=debianred}

\setbeamercolor{block title}{fg=debianblue}

\title[Reproducible Builds]{Reproducible Builds}

\author[lamby]{%
   \texorpdfstring{
            \centering
            Chris Lamb \\ 
            \href{mailto:lamby@debian.org}{lamby@debian.org}
   }{lamby}}
\institute[Debian]{}
\date[skroutz.gr '15]{%
 skroutz.gr (Athens, Greece)\\
 \small{2015-12-10}}

\begin{document}

\begin{frame}
 \titlepage
\end{frame}

\begin{frame}
 \frametitle{Debian reproducible builds team}
 \begin{center}
  \begin{columns}
   \small
   \column{.33\linewidth}
    {akira} \\
    {Andrew Ayer} \\
    {Asheesh Laroia} \\
    \only<1>{Chris Lamb}\only<2>{{\color{debianblue} Chris Lamb}} \\
    {Chris West} \\
    {Christoph Berg} \\
    {Daniel Kahn Gillmor} \\
    David Suarez \\
    {Dhole} \\
    Drew Fisher \\
    Esa Peuha \\
    {Guillem Jover} \\
   \column{.33\linewidth}
    Hans-Christoph Steiner \\
    {Helmut Grohne} \\
    {Holger Levsen} \\
    {Jelmer Vernooij} \\
    {josch} \\
    Juan Picca \\
    {Lunar} \\
    Mathieu Bridon \\
    {Mattia Rizzolo} \\
    Nicolas Boulenguez \\
    {Niels Thykier} \\
    Niko Tyni \\
   \column{.33\linewidth}
    {Paul Wise} \\
    Peter De Wachter \\
    Philip Rinn \\
    {Reiner Herrmann} \\
    {Stefano Rivera} \\
    {Stéphane Glondu} \\
    {Steven Chamberlain} \\
    Tom Fitzhenry \\
    Valentin Lorentz \\
    {Wookey} \\
    {Ximin Luo} \\
  \end{columns}
 \end{center}
\end{frame}

\begin{frame}
 \frametitle{Who are you?}
 \begin{itemize}
  \item Contributed to Free Software?
  \item<2-3> Seen a talk about reproducible builds this year?
  \item<3> Contributed to this effort?
 \end{itemize}
\end{frame}

\section{About}

\begin{frame}
 \frametitle{The problem}

 \begin{center}
  \includegraphics[width=0.7\textwidth]{images/31c3.png}

  Available on \url{media.ccc.de}, 31c3
 \end{center}
\end{frame}

\begin{frame}[fragile]
 \frametitle{A few example's from that 31c3 talk}
 \begin{itemize}
  \item CVE-2002-0083: remote root exploit in \texttt{sshd}, a single bit difference in binary
  \item 31c3 talk: live demo with kernel module modifying source code in memory only
  \item financial incentives to crack developer machines…
  \item how can you be sure what's running on your machine or on a build
  daemon network? Do you ever leave your USB3 ports alone?
 \end{itemize}
\end{frame}

\begin{frame}[fragile]
 \frametitle{Another example from real life}

 At a CIA conference in 2012:
 \begin{center}
  \includegraphics[width=0.8\textwidth]{images/strawhorse.png}

  {\footnotesize
  \url{firstlook.org/theintercept/2015/03/10/ispy-cia-campaign-steal-apples-secrets/}
  }
 \end{center}
\end{frame}


\begin{frame}
 \frametitle{The solution}

 \begin{center}
 \Large{
 Promise that anyone can always generate
 identical binary packages
 from a given source}
\end{center}
\end{frame}


\begin{frame}
 \frametitle{The solution}

 \begin{center}
 We call this:

 \Huge{ “Reproducible builds” }
 \end{center}
\end{frame}

\section{Progress}

\begin{frame}[plain]
 \frametitle{Progress in Debian \texttt{unstable}}
 \begin{center}
  \includegraphics[height=0.73\paperheight]{images/stats_pkg_state.png}

  \footnotesize{19,257 out of 23,141 source packages are reproducible \\
    in our test framework}
  \vfill
 \end{center}
\end{frame}

\begin{frame}
 \frametitle{What we did in Debian since Summer 2014}

 \begin{itemize}
  \item Agreed on using a fixed build path: \texttt{/build/}
  \item Recording the build environment: \texttt{.buildinfo}
  \item \texttt{strip-nondeterminism}
  \item \texttt{reproducible.debian.net}
  \item \texttt{diffoscope} (formerly \texttt{debbindiff})
  \item \texttt{SOURCE\_DATE\_EPOCH}
  \item \texttt{disorderfs}
  \item 700+ patches: \texttt{dpkg}, \texttt{debhelper}, \texttt{sbuild}, …
  \item<2> Tell the world \& collaborate
 \end{itemize}
\end{frame}


\begin{frame}
 \frametitle{Tell the world \& collaborate}

 \begin{itemize}
  \item Recent talks (some available with subtitles):
   \begin{itemize}
    \item 2015-08-13: Chaos Communication Camp 2015
    \item 2015-08-20: DebConf15
    \item 2015-11-08: Mini-DebConf Cambridge 2015
   \end{itemize}
  \item Weekly reports since May 2015
  \item Summit in December 2015 (Athens)
   \begin{itemize}
    \item 40 people from 16 projects
   \end{itemize}
 \end{itemize}
\end{frame}

\begin{frame}
 \frametitle{Tell the world \& collaborate, cont.}

 \begin{itemize}
  \item \texttt{https://reproducible-builds.org}
 \end{itemize}
 \begin{center}
 \includegraphics[width=0.7\textwidth]{images/rbwww1.png}
 \end{center}
\end{frame}

\begin{frame}
 \frametitle{Stats about reproducible.debian.net}

 \begin{itemize}
  \item Continuously testing Debian testing, unstable and experimental
   \begin{itemize}
    \item \small{ \texttt main only }
    \item \small{ can we build \texttt contrib without legal troubles? }
   \end{itemize}
  \item Also testing coreboot, OpenWrt, NetBSD, FreeBSD,
  Archlinux and soon Fedora
   \begin{itemize}
    \item \small{ those currently only weekly though… }
   \end{itemize}
 \end{itemize}
 \vfill
 \begin{center}
  \includegraphics[height=0.47\paperheight]{images/stats_builds_per_day_amd64.png}
 \end{center}
\end{frame}


\begin{frame}
 \frametitle{More stats on reproducible.debian.net}

 \begin{itemize}
  \item 122 jenkins jobs running on 12 hosts
  \item 27 contributors for \texttt{jenkins.debian.net.git}
  \item 4k lines of Python and 5k lines Bash code
  \item \texttt{amd64}: 111 cores and 198 GB RAM split on 9 VMs, provided by
  https://profitbricks.co.uk
  \item \texttt{armhf}: 18 cores and 9 GB RAM on 6 systems, provided by vagrant@d.o.
 \end{itemize}
 \begin{center}
  \includegraphics[height=0.2\paperheight]{images/profitbricks_logo.png}
  \vfill
 \end{center}
\end{frame}

\begin{frame}
 \frametitle{Good to know about reproducible.debian.net}

 \begin{itemize}
  \item \url {https://reproducible.debian.net/$src}
  \item<2-3> { 165 categorised distinct issues }
  \item<2-3> { 3,496 packages to be fixed in \texttt{sid}, but only 426 without annotated
  issues }
  \item<3> { 29 different "package sets", eg. \texttt{build-essential} is only 78\%
  reproducible
   \begin{center}
    \includegraphics[height=0.5\paperheight]{images/stats_meta_pkg_state_build-essential.png}
   \vfill
 \end{center}
  }
 \end{itemize}
\end{frame}


\begin{frame}[fragile]
 \frametitle{Variations on reproducible.debian.net}

 \begin{center}
  \begin{table}
   \resizebox{0.95\textwidth}{!}{%
    \begin{tabular}{l|ll}
\textbf{variation} & \textbf{first build} & \textbf{second build} \\
\hline
hostname & \texttt{jenkins} & \texttt{i-capture-the-hostname} \\
domainname & \texttt{debian.net} & \texttt{i-capture-the-domainname} \\
\texttt{env TZ} & \texttt{GMT+12} & \texttt{GMT-14} \\
\texttt{env LANG} & \texttt{C} & \texttt{fr\_CH.UTF-8} \\
\texttt{env LC\_ALL} & not set & \texttt{fr\_CH.UTF-8} \\
\texttt{env USER} & \texttt{pbuilder1} & \texttt{pbuilder2} \\
uid & \texttt{1111} & \texttt{2222} \\
gid & \texttt{1111} & \texttt{2222} \\
UTS namespace & shared with the host & \textit{modified using \texttt{/usr/bin/unshare --uts}} \\
kernel version & Linux 3.16.0-4 / 4.2.0-0.bpo & Linux 2.6.56-4 / 2.6.62-0.bpo.1 \\
umask & 0022 & 0002 \\
CPU type & \multicolumn{2}{l}{same for both builds \textit{(work in progress)}} \\
filesystem & \multicolumn{2}{l}{same for both builds \textit{(work in progress - disorderfs)}} \\
year, month, date & \multicolumn{2}{l}{same for both builds \textit{(work in progress)}} \\
hour, minute & \multicolumn{2}{l}{hour is usually the same… usually, the minute differs… \textit{(work in progress)}} \\
\textit{everything else} & \multicolumn{2}{l}{\textit{is likely the same…}}
    \end{tabular}
   }
  \end{table}
 \end{center}
\end{frame}


{
\usebackgroundtemplate{%
 \begin{tikzpicture}[remember picture,overlay]%
  \node[shift={(-0.15\paperwidth, 0.4\paperheight)},at=(current page.south east)] {
    \includegraphics[width=0.2\paperwidth]{images/diffoscope_logo.png}
  };
 \end{tikzpicture}%
}
\begin{frame}{diffoscope}
 \frametitle{Debugging problems: diffoscope}

 \begin{itemize}
  \item Examines differences \textbf{in depth}.
  \item Outputs HTML or plain text with human readable differences.
  \item Recursively unpacks archives, uncompresses PDFs, disassembles
  binaries, unpacks Gettext files, …
  \item Easy to extend to new file formats.
  \item Falls back to binary comparison.
  \item Available from \texttt{git}, PyPI, Debian (sid and stretch), \\
   Arch Linux, Guix, Homebrew.
  \item Maintainers in other distros wanted.
  \item \url{http://diffoscope.org/}
 \end{itemize}
\end{frame}
}

\begin{frame}
 \frametitle{diffoscope example (HTML output)}
 \begin{tikzpicture}[remember picture]
  \node[at=(current page.center)] {
   \includegraphics[width=0.9\paperwidth]{images/diffoscope_example_html.png}
  };
 \end{tikzpicture}
\end{frame}

\begin{frame}
 \frametitle{\texttt{SOURCE\_DATE\_EPOCH}}

 \begin{itemize}
  \item Build date usually not useful for the user
  \item Value of \texttt{SOURCE\_DATE\_EPOCH} instead of current date \& for other seeds
  \item In Debian, set from the latest \texttt{debian/changelog} entry
  \item General solution for other projects \& distributions
 \end{itemize}
\end{frame}

\section{Beyond building}

\begin{frame}
 \frametitle{Reproducible builds demand a defined build environment}
 \begin{itemize}
  \item Re-creating an identical build environment is mandatory too.
  \item Without an identical build environment, reproducible builds will only
  happen by sheer luck.
  \item<2>{Only solved for Debian right now and currently proof of concept only…}
 \end{itemize}
\end{frame}

\begin{frame}
 \frametitle{Debian release process}
 \begin{itemize}
  \item In our current design and practices, rebuilding stretch will require
  package versions which are not part of stretch.
  \item This design might put a high load on snapshot.debian.org.
  \item<2-4>{Rebuilding all of Debian a month prio the release? }
  \item<3-4>{Cross-builds could even speed up slow archs.}
  \item<4>{More discussions needed. Freeze probably on November 5th 2016.}
 \end{itemize}
\end{frame}

\begin{frame}
 \frametitle{Distributing \texttt{.buildinfo} files}
 \begin{itemize}
  \item Probably 100,000 new files per Debian suite; 50\% increase per suite
  \item Mirrors would not be happy, so should not go there
  \item We'll need more files when we have detached signatures
  \item<2>{Revoking signatures?}
  \item<2>{...}
 \end{itemize}
\end{frame}

\begin{frame}
 \frametitle{Rebuilders and sharing signed checksums}
 \begin{itemize}
  \item Almost no work has been done here yet.
  \item<2-3> Continuous rebuilds should happen in a systematic way and resulting
  checksums properly published.
  \item<3> And then we need a system to sign those checksums and share them. 
 \end{itemize}
\end{frame}

\begin{frame}
 \frametitle{Rebuilders and sharing signed checksums, cont.}
 \begin{itemize}
  \item Individuelly signed checksums (think web of trust) could work in the
  Debian case (we have a gpg web of trust), but won't scale.
  \item<2-4> { We'll probably could use systematic rebuilders, run by large organisations
  (ACLU, CCC, CERN, DECIX, DESY, Deutsche Bank, EDF, EON, Greenpeace, NASA, NSA, XYZ).}
  \item<3-4> { …and automated installers for those… }
  \item<4> { …and howtos (\texttt {gpg --gen-key})…}
 \end{itemize}
\end{frame}


\section{Want to help?}

\begin{frame}
 \frametitle{As a developer}
 \begin{itemize}
  \item Stop using build dates
  \item Use \texttt{SOURCE\_DATE\_EPOCH} instead
  \item See \url{https://reproducible-builds.org/specs/}
 \end{itemize}
\end{frame}

\begin{frame}
 \frametitle{Get involved - learning by doing}

 \begin{itemize}
  \item Test for yourself:
   \begin{itemize}
    \item Build something twice, run diffoscope on the results
    \begin{itemize}
     \item For better results use our “reproducible” repository, \texttt{pbuilder} and a custom config
    \end{itemize}
   \end{itemize}
  \item Docs on the web: \\
    \small{\url{https://reproducible-builds.org/docs/}} \\
    \small{\url{https://wiki.debian.org/ReproducibleBuilds/ExperimentalToolchain}}
  \item Ask for help on \texttt{\#debian-reproducible} or on mailing list
 \end{itemize}
\end{frame}

\begin{frame}
 \frametitle{Join the team!}

 \begin{itemize}
  \item Why?
   \begin{itemize}
    \item \heartsuit{}\heartsuit{}\heartsuit{} Lovely group of people \heartsuit{}\heartsuit{}\heartsuit{}
    \item Learn something new everyday
    \item Change the (software) world!
   \end{itemize}
  \item What do we do?
   \begin{itemize}
    \item Review packages
    \item Identify issues and document solutions
    \item \texttt{reproducible.d.n}, diffoscope, strip-nondeterminism
    \item Propose changes for toolchain
    \item Submit patches for individual packages
    \item Write more general documentation and talk to the world
   \end{itemize}
 \end{itemize}
\end{frame}

\begin{frame}
 \frametitle{Create a new team!}

 \begin{itemize}
  \item Why?
   \begin{itemize}
    \item Every distribution should be reproducible!
    \item Learn something new everyday
    \item Change the (software) world!
   \end{itemize}
  \item How to get started?
   \begin{itemize}
    \item Talk to me here or talk to us on IRC or via mail.
    \item RTFM, there is lots of documentation
    \item Experiment - learning by doing
   \end{itemize}
 \end{itemize}
\end{frame}

\section{Questions, comments, ideas?}

\begin{frame}
 \frametitle{Questions, comments, ideas?}

 \begin{itemize}
  \item \url{https://reproducible-builds.org}
  \item \url{https://reproducible.debian.net}
  \item \texttt{\#debian-reproducible} on \texttt{irc.OFTC.net}
 \end{itemize}
\end{frame}


\begin{frame}
 \frametitle{Thanks!}

 \begin{itemize}
  \item Debian “Reproducible Builds” team \\
        {\small (you are just \textbf{so} awesome!)}
  \item Linux Foundation and the Core Infrastructure Initiative
\end{itemize}

 \begin{center}
  \includegraphics[height=0.1\paperheight]{images/linux_foundation_logo.png}
  \hspace{0.1\paperwidth}
  \includegraphics[height=0.1\paperheight]{images/cii_logo.png}
 \end{center}

 \vfill
 \begin{center}
  \resizebox{0.8\textwidth}{!}{%
   \begin{tabular}{rl}
    \texttt{lamby@debian.org} & \texttt{C2FE 4BD2 71C1 39B8 6C53} \\
                              & \texttt{3E46 1E95 3E27 D431 1E58} 
   \end{tabular}
  }
 \end{center}
\end{frame}

\end{document}
