\documentclass[14pt]{beamer}
\setbeamertemplate{caption}[numbered]
\setbeamertemplate{caption label separator}{:}
\setbeamercolor{caption name}{fg=normal text.fg}
\usepackage{amssymb,amsmath}
\usepackage{ifxetex,ifluatex}
\usepackage{fixltx2e} % provides \textsubscript
\usepackage{lmodern}
\ifxetex
  \usepackage{fontspec,xltxtra,xunicode}
  \defaultfontfeatures{Mapping=tex-text,Scale=MatchLowercase}
  \newcommand{\euro}{€}
\else
  \ifluatex
    \usepackage{fontspec}
    \defaultfontfeatures{Mapping=tex-text,Scale=MatchLowercase}
    \newcommand{\euro}{€}
  \else
    \usepackage[T1]{fontenc}
    \usepackage[utf8]{inputenc}
      \fi
\fi
% use upquote if available, for straight quotes in verbatim environments
\IfFileExists{upquote.sty}{\usepackage{upquote}}{}
% use microtype if available
\IfFileExists{microtype.sty}{\usepackage{microtype}}{}
\PassOptionsToPackage{hyphens}{url}
\usepackage{hyperref}
\usepackage{ulem}

% Comment these out if you don't want a slide with just the
% part/section/subsection/subsubsection title:
\AtBeginPart{
  \let\insertpartnumber\relax
  \let\partname\relax
  \frame{\partpage}
}
\AtBeginSection{
  \let\insertsectionnumber\relax
  \let\sectionname\relax
  \begin{frame}[plain]
    \tableofcontents[currentsection]
  \end{frame}
}
\AtBeginSubsection{
  \let\insertsubsectionnumber\relax
  \let\subsectionname\relax
  \frame{\subsectionpage}
}

\setlength{\parindent}{0pt}
\setlength{\parskip}{6pt plus 2pt minus 1pt}
\setlength{\emergencystretch}{3em}  % prevent overfull lines
\setcounter{secnumdepth}{0}
% Thanks Richard Darst on how to get a nice Beamer theme.
% See http://rkd.zgib.net/wiki/DebianBeamerThemes

\usepackage{multicol}
\usepackage[absolute,overlay]{textpos}
\usepackage{tikz}
\usepackage{ctable}
\usetikzlibrary{positioning}

\usebackgroundtemplate{\includegraphics[width=\paperwidth]{images/swirl-lightest.pdf}}
\newif\ifplacelogo
\placelogotrue
\logo{\ifplacelogo\includegraphics[viewport=274 335 360 440,width=1cm]{images/openlogo-nd.pdf}\fi}

\definecolor{debianred}{rgb}{.780,.000,.211} % 199,0,54
\definecolor{debianblue}{rgb}{0,.208,.780} % 0,53,199
\definecolor{debianlightbackgroundblue}{rgb}{.941,.941,.957} % 240,240,244
\definecolor{debianbackgroundblue}{rgb}{.776,.784,.878} % 198,200,224

\usetheme{Boadilla}
\setbeamertemplate{navigation symbols}{}

\usecolortheme[named=debianbackgroundblue]{structure}
\setbeamercolor{normal text}{fg=black}
\setbeamercolor{titlelike}{fg=debianblue}
\setbeamercolor{sidebar}{fg=debianred,bg=debianbackgroundblue}

\setbeamercolor{palette sidebar primary}{fg=debianred}
\setbeamercolor{palette sidebar secondary}{fg=debianred}
\setbeamercolor{palette sidebar tertiary}{fg=debianred}
\setbeamercolor{palette sidebar quaternary}{fg=debianred}

\setbeamercolor{section in toc}{fg=debianred}
\setbeamercolor{subsection in toc}{parent=debianred}

\setbeamercolor{item}{fg=debianred}

\setbeamercolor{block title}{fg=debianblue}

\title[Reproducible builds ecosystem]{Reproducible builds ecosystem}
\subtitle{Where some of us are \\
and some hints where this might be going…}
\author[Holger 'h01ger' Levsen]{%
   \texorpdfstring{
            \centering
            Holger 'h01ger' Levsen\\
            \href{mailto:holger@layer-acht.org}{holger@layer-acht.org}
   }{h01ger}}
\institute[Debian]{}
\date[ripe72]{%
 ripe72 (Copenhagen, DK)\\
 \small{2016-05-24}}

\begin{document}

\begin{frame}[plain]
 \titlepage
\end{frame}

\begin{frame}
 \frametitle{about me}

 \begin{itemize}
  \item \small{\texttt{B8BF 5413 7B09 D35C F026  FE9D 091A B856 069A AA1C}}
  \item Debian user since 1995
  \item Debian contributor since 2001
  \item Debian developer since 2007
  \item DebConf organizer,
  founded the DebConf video team
   \begin{itemize}
    \item \texttt{http://video.debian.net}
   \end{itemize}
 \item Debian-Edu (Debian for education)
  \item Debian QA (quality assurance)
  \begin{itemize}
   \item \texttt{https://piuparts.debian.org}
   \item \texttt{https://jenkins.debian.net} (~1100 jobs continously testing Debian)
  \end{itemize}
  \item \sout{Debian LTS} (Long Term Support)
  \item \texttt{sudo (apt|dnf) install torbrowser-launcher}
 \end{itemize}
\end{frame}


\begin{frame}
 \frametitle{more about me}

 \begin{itemize}
  \item \small{\texttt{B8BF 5413 7B09 D35C F026  FE9D 091A B856 069A AA1C}}
  \item \small{\texttt{8F03 B243 8719 BA6B 1A35  0EB6 40C2 DEA2 F56C 7256}}
  \item Debian Reproducible builds team member
  \begin{itemize}
   \item until April 2016 together with Lunar funded by the Linux Foundation
   \item within in the team I'm mostly working on
   \texttt{https://tests.reproducible-builds.org}
   \item applied for extended funding in April 2016…
  \end{itemize}
\end{itemize}
\end{frame}

\begin{frame}
 \frametitle{Debian reproducible builds team}
 \begin{center}
  \begin{columns}
   \small
   \column{.30\linewidth}
    {akira} \\
    {Alexis Bienvenüe} \\
    {Andrew Ayer} \\
    {Asheesh Laroia} \\
    {Ceridwen} \\
    {Chris Lamb} \\
    {Chris West} \\
    {Christoph Berg} \\
    {Daniel Kahn Gillmor} \\
    {Daniel Shahaf} \\
    David Suarez \\
    {Dhole} \\
    Drew Fisher \\
    Esa Peuha \\
    {Fabian Wolff} \\
   \column{.30\linewidth}
    {Guillem Jover} \\
    Hans-Christoph Steiner \\
    {Helmut Grohne} \\
    \only<1>{Holger Levsen}\only<2>{{\color{debianred} Holger Levsen}} \\
    {Jelmer Vernooij} \\
    {josch} \\
    Juan Picca \\
    {Lunar} \\
    Mathieu Bridon \\
    {Mattia Rizzolo} \\
    Nicolas Boulenguez \\
    {Niels Thykier} \\
    Niko Tyni \\
    {Paul Wise} \\
   \column{.30\linewidth}
    Peter De Wachter \\
    Philip Rinn \\
    {Reiner Herrmann} \\
    {Sascha Steinbiss} \\
    {Satyam Zode} \\
    {Scarlett Clark} \\
    {Stefano Rivera} \\
    {Stéphane Glondu} \\
    {Steven Chamberlain} \\
    Tom Fitzhenry \\
    {Valerie Young} \\
    Valentin Lorentz \\
    {Wookey} \\
    {Ximin Luo} \\
  \end{columns}
 \end{center}
\end{frame}


\begin{frame}
 \frametitle{jenkins.debian.net.git contributors}
 \begin{center}
  \begin{columns}
   \small
   \column{.46\linewidth}
    {akira} \\
    \only<1>{Alexander Couzens}\only<2>{{\color{debianred} Alexander Couzens}} \\
    \only<1>{Levente 'anthraxx' Polyak}\only<2>{{\color{debianred} Levente 'anthraxx' Polyak}} \\
    {Antonio Terceiro} \\
    {Axel Beckert} \\
    \only<1>{Bryan Newbold}\only<2>{{\color{debianred} Bryan Newbold}} \\
    {Chris Lamb} \\
    {Daniel Kahn Gillmor} \\
    {Gabriele Giacone} \\
    \only<1>{Hans-Christoph Steiner}\only<2>{{\color{debianred} Hans-Christoph Steiner}} \\
    Helmut Grohne \\
    \only<1>{Holger Levsen}\only<2>{{\color{debianred} Holger Levsen}} \\
    \only<1>{HW42}\only<2>{{\color{debianred} HW42}} \\
    {James McCoy} \\
    {Joachim Breitner} \\
   \column{.46\linewidth}
    {Johannes 'josch' Schauer} \\
    {Jérémy Bobbio} \\
    {Mattia Rizzolo} \\
    {Niels Thykier} \\
    {Paul Wise} \\
    {Petter Reinholdtsen} \\
    {Philip Hands} \\
    \only<1>{Reiner Herrmann}\only<2>{{\color{debianred} Reiner Herrmann}} \\
    {Samuel Thibault} \\
    {Steven Chamberlain} \\
    {Tails developers} \\
    {Ulrike Uhlig} \\
    {Wolfgang Schweer} \\
    {Wouter Verhelst} \\
  \end{columns}
 \end{center}
\end{frame}


\begin{frame}
 \frametitle{Who are you?}
 \begin{itemize}
  \item Contributed to Free Software?
  \item<2> Seen a talk about reproducible builds?
 \end{itemize}
\end{frame}

\section{Motivation}

\begin{frame}
 \frametitle{The problem}

 \begin{center}
  \includegraphics[width=0.7\textwidth]{images/31c3.png}

  Available on \url{media.ccc.de}, 31c3
 \end{center}
\end{frame}

\begin{frame}[fragile]
 \frametitle{A few examples from that 31c3 talk}
 \begin{itemize}
  \item CVE-2002-0083: remote root exploit in \texttt{sshd}, a single bit difference in the binary
  \item<2-5> 31c3 talk had a live demo with a kernel module modifying source code in memory only
  \item<3-5> financial incentives to crack developer machines…
  \item<4-5> {how can you be sure what's running on your machine or on a build
  daemon network? Do you ever leave your} \only<4>{USB3 ports alone?}\only<5>{computers alone?}
 \end{itemize}
\end{frame}

\begin{frame}[fragile]
 \frametitle{Another example from real life}

 At a CIA conference in 2012:
 \begin{center}
  \includegraphics[width=0.8\textwidth]{images/strawhorse.png}

  {\footnotesize
  \url{firstlook.org/theintercept/2015/03/10/ispy-cia-campaign-steal-apples-secrets/}
  }
 \end{center}
\end{frame}


\begin{frame}
 \frametitle{The solution}

 \begin{center}
 \Large{
 Promise that anyone can always generate
 identical binary packages
 from a given source}
\end{center}
\end{frame}


\begin{frame}
 \frametitle{The solution}

 \begin{center}
 We call this:

 \Huge{ “Reproducible builds” }
 \end{center}
\end{frame}

\begin{frame}
 \frametitle{Demo}
% show this once running in plain sid,
% and then in sid with our modified toolchain.
%
% prepare demo:
% mkdir demo ; cd demo ; apt-get source giftrans
%
% do demo:
% PTH=$(mktemp -d); OPTH=$PWD; P=giftrans; cp ${P}_* $PTH/; cd $PTH ;
%   dpkg-source -x ${P}*.dsc ; for X in 1 2 3 4 5 ; do (cd ${P}-*/;
%   dpkg-buildpackage -b -uc -us); mkdir -p .$X ; cp $P_*.deb .$X; done ; rm
%   *.deb ; echo; sha1sum *dsc *z .*/*.deb | grep -v giftrans-dbgsym ; cd - ;
% rm -r $PTH
\end{frame}

\begin{frame}[plain]
\begin{center}
 \Huge{This should become the \textbf{norm}.}

 \visible<2>{\small{ We want to change the meaning of "free software":

  it's only free software if it's reproducible!}}
\end{center}
\end{frame}


\section{Common ressources}

\begin{frame}
 \frametitle{reproducible-builds.org}

 \begin{itemize}
  \item \texttt{https://reproducible-builds.org}
 \end{itemize}
 \begin{center}
 \includegraphics[width=0.7\textwidth]{images/rbwww1.png}
 \end{center}
\end{frame}

\begin{frame}
 \frametitle{tests.reproducible-builds.org}

 \begin{itemize}
  \item<2-5> Continuously testing Debian \texttt{testing}, \texttt{unstable} and
  \texttt{experimental}
  \item<3-5> on \texttt{amd64} and \texttt{i386} and \texttt{armhf}
  \item<4-5> Also testing: coreboot, OpenWrt, NetBSD, FreeBSD,
  Arch Linux, Fedora and soon F-Droid and Guix too
  \item<5> 283 jenkins jobs running on 28 hosts
  \item<5> 41 scripts with a total of 4k lines of Python and 6k lines of Bash
  Shell
  \item<5> 31 contributors for \texttt{jenkins.debian.net.git}
 \end{itemize}
\end{frame}


\begin{frame}
 \frametitle{CPU architectures on tests.r-b.org}

 \begin{itemize}
  \item \texttt{amd64} and \texttt{i386}: 106 cores and 282 GB RAM split on 9 VMs
  \item most ressources used for testing Debian…
 \item sponsored by \texttt{https://profitbricks.co.uk} since 2014 (2012)
 \end{itemize}
 \begin{center}
  \includegraphics[height=0.2\paperheight]{images/profitbricks_logo.png}
  \vfill
 \end{center}
 \begin{itemize}
  \item \texttt{armhf}: 18 nodes with 70 cores and 35 GB RAM sponsored by
  Debian
  \item \texttt{arm64}: coming soon
 \end{itemize}
\end{frame}


\begin{frame}[fragile]
 \frametitle{Variations (when testing Debian)}

 \begin{center}
  \begin{table}
   \resizebox{0.97\textwidth}{!}{%
    \begin{tabular}{l|ll}
\textbf{variation} & \textbf{first build} & \textbf{second build} \\
\hline
hostname & \texttt{jenkins} & \texttt{i-capture-the-hostname} \\
domainname & \texttt{debian.net} & \texttt{i-capture-the-domainname} \\
\texttt{env TZ} & \texttt{GMT+12} & \texttt{GMT-14} \\
\texttt{env LANG} & \texttt{C} & \texttt{fr\_CH.UTF-8} \\
\texttt{env LC\_ALL} & not set & \texttt{fr\_CH.UTF-8} \\
\texttt{env USER} & \texttt{pbuilder1} & \texttt{pbuilder2} \\
uid & \texttt{1111} & \texttt{2222} \\
gid & \texttt{1111} & \texttt{2222} \\
shell & \texttt{dash} & \texttt{bash} \\
UTS namespace & shared with the host & \textit{modified using \texttt{/usr/bin/unshare --uts}} \\
kernel version & Linux 3.16 or 4.X & on amd64 always varied, on armhf
sometimes \\
umask & 0022 & 0002 \\
CPU type & \multicolumn{2}{l}{same for both builds on amd64 \textit{(work in progress)}} \\
 & on armhf varied a bit & \\
filesystem & \multicolumn{2}{l}{same for both builds on amd64: (\texttt{tmpfs}), on armhf \texttt{ext3/4}} \\
 & & \textit{(and we have} \texttt{disorderfs}\textit{, but the code is disabled)} \\
year, month, date & \multicolumn{2}{l}{on amd64: 398 days variation, on armhf not yet} \\
hour, minute & \multicolumn{2}{l}{hour is usually the same… usually, the minute differs… } \\
\textit{everything else} & \multicolumn{2}{l}{\textit{is likely the same…}}
    \end{tabular}
   }
  \end{table}
 \end{center}
\end{frame}




\begin{frame}
 \frametitle{Common problems}

 \begin{itemize}
  \item time stamps
  \item<2-3> timezones
  \item<2-3> locales
  \item<3> everything else (seperated into known issues and the blurry rest)
 \end{itemize}
\end{frame}

\begin{frame}
 \frametitle{Documentation about common problems}
 \begin{itemize}
  \item \texttt{https://reproducible-builds.org/docs}
  \item Lunar's talk from CCCamp 2015 also on
  \texttt{https://media.ccc.de}
 \begin{tikzpicture}[remember picture]
  \node[shift={(-1.05\paperwidth, -0.3\paperheight)},at=(current page.south east)] {
    \includegraphics[width=0.83\textwidth]{images/cccamp2015_lunar_random.png}
  };
 \end{tikzpicture}
 \end{itemize}
\end{frame}



\placelogofalse

{
\usebackgroundtemplate{%
 \begin{tikzpicture}[remember picture,overlay]%
  \node[shift={(-0.1\paperwidth, 0.15\paperheight)},at=(current page.south east)] {
    \includegraphics[width=0.2\paperwidth]{images/diffoscope_logo.png}
  };
 \end{tikzpicture}%
}

\begin{frame}{diffoscope}
 \frametitle{Debugging problems: \texttt{diffoscope}}

 \begin{itemize}
  \item Examines differences \textbf{in depth}.
  \item Outputs HTML or plain text with human readable differences.
  \item Recursively unpacks archives, uncompresses PDFs, disassembles
  binaries, unpacks Gettext files, …
  \item Easy to extend to new file formats.
  \item Falls back to binary comparison.
  \item Available from \texttt{git}, PyPI, Debian (sid and stretch), \\
   Fedora, Arch Linux, FreeBSD, NetBSD, Guix, Homebrew..
  \item Maintainers (upstream and in other distros) wanted.
  \item \url{https://diffoscope.org/}
 \end{itemize}
\end{frame}


\begin{frame}
 \frametitle{\texttt{diffoscope} example (HTML output)}
 \begin{tikzpicture}[remember picture]
  \node[at=(current page.center)] {
   \includegraphics[width=0.9\paperwidth]{images/diffoscope_example_html.png}
  };
 \end{tikzpicture}
\end{frame}


\begin{frame}{diffoscope}
 \frametitle{Try \texttt{diffoscope}}
 \begin{itemize}
  \item \texttt{https://try.diffoscope.org}
 \end{itemize}
\end{frame}



\begin{frame}
 \frametitle{\texttt{diffoscope} is "just" for debugging}

 \begin{itemize}
  \item Reminder: \texttt{diffoscope} is for \textbf{debugging}
  \item<2> "reproducible" according to our definition means: \textbf{bit by bit
  identical}. So the tools for testing whether something is reproducible are
  either \texttt{diff} or \texttt{sha256sum}!
 \end{itemize}
\end{frame}

}

\placelogotrue


\begin{frame}
 \frametitle{\texttt{SOURCE\_DATE\_EPOCH}}

 \begin{itemize}
  \item Build date (timestamps) usually not useful for the user
  \item<2-3> \texttt{SOURCE\_DATE\_EPOCH} is defined as the last modification of
  the source, since the epoch (1970-01-01)
  \item<2-3> \texttt{SOURCE\_DATE\_EPOCH} can be used instead of current date
  \item<2-3> can also be used for random seeds etc.
  \item<3> in Debian, set from the latest \texttt{debian/changelog} entry
  \item<3> solution has been adopted by other projects \& distributions
  (NetBSD, FreeBSD, Arch Linux, Guix, Fedora…)
 \end{itemize}
\end{frame}


\begin{frame}
 \frametitle{\texttt{SOURCE\_DATE\_EPOCH} (closed bugs)}

 \begin{itemize}
  \item dh-strip-nondeterminism
  \item gcc (\texttt{\_\_DATE\_\_} and \texttt{\_\_TIME\_\_} macros) \texttt{\footnotesize{\url{https://gcc.gnu.org/ml/gcc-patches/2015-06/msg02210.html}}}
  \item \sout{\texttt{\#791823}}: debhelper
  \item \sout{\texttt{\#787444}}: help2man
  \item \sout{\texttt{\#790899}}: epydoc
  \item \sout{\texttt{\#794004}}: ghostscript
  \item \sout{\texttt{\#796130}}: man2html
  \item \sout{\texttt{\#783475}}: texi2html
  \item \sout{\texttt{\#794586}}: ocamldoc
  \item \sout{\texttt{\#795942}}: wheel
  \item \sout{\texttt{\#792202}}: texlive-bin
  \item ...
 \end{itemize}
\end{frame}


\begin{frame}
 \frametitle{\texttt{SOURCE\_DATE\_EPOCH} (open/pending bugs)}

 \begin{itemize}
  \item \texttt{\#792687, \#804141}: gettext
  \item \texttt{\#792201}: doxygen
  \item \texttt{\#800797}: docbook-utils
  \item \texttt{\#801621}: perl (Pod::Man)
  \item \texttt{\#790801}: txt2man
  \item \texttt{\#791815}: libxslt
  \item \texttt{\#794681}: qt4-x11 (qthelpgenerator)
  \item ...
 \end{itemize}

\end{frame}

\begin{frame}
 \frametitle{\texttt{SOURCE\_DATE\_EPOCH}}

 \begin{itemize}
  \item \texttt{SOURCE\_DATE\_EPOCH} spec availble
  \item \texttt{https://reproducible-builds.org/specs/}
 \end{itemize}
\end{frame}


\section{Status Debian}

\begin{frame}
 \frametitle{Progress in Debian \texttt{unstable}}
 \begin{tikzpicture}[remember picture]
  \node[shift={(-0.75\paperwidth, -0.3\paperheight)},at=(current page.south east)] {
    \includegraphics[height=0.65\paperheight]{images/stats_pkg_state.png}
  };
 \end{tikzpicture}
 \begin{center}
  \footnotesize{21,365 (88.5\%) out of 24,135 source packages are reproducible \\
    in our test framework}
  \vfill
 \end{center}
\end{frame}

\begin{frame}
 \frametitle{Notes and issues on tests.reproducible-builds.org}

 \begin{itemize}
  \item { 194 categorised distinct issues }
  \item { 3,085 notes }
  \item<2-4> { 1879 unreproducible packages in \texttt{sid/amd64}, but only
  227 without a  note }
  \item<2-4> { 516 packages failing to build, but only 92 without a note }
  \item<3-4> { maintained in \texttt{notes.git} }
  \item<4> { currently Debian only, but cross distro notes are planned}
 \end{itemize}
\end{frame}



\begin{frame}
 \frametitle{Debian packages on tests.reproducible-builds.org}
 \begin{itemize}
  \item \url {https://reproducible.debian.net/$src}
 \end{itemize}
\end{frame}

\begin{frame}
 \frametitle{Debian package sets on tests.r-b.org}
 \begin{tikzpicture}[remember picture]
  \node[shift={(-0.75\paperwidth, -0.3\paperheight)},at=(current page.south east)] {
    \includegraphics[height=0.65\paperheight]{images/stats_meta_pkg_state_required.png}
  };
 \end{tikzpicture}
 \begin{center}
  \footnotesize{35 different "package sets", eg. \texttt{required} is only
  70.2\% reproducible}
  \vfill
 \end{center}
\end{frame}

\begin{frame}
 \frametitle{Debian package sets on tests.r-b.org}
 \begin{tikzpicture}[remember picture]
  \node[shift={(-0.75\paperwidth, -0.3\paperheight)},at=(current page.south east)] {
    \includegraphics[height=0.65\paperheight]{images/stats_meta_pkg_state_key_packages.png}
  };
 \end{tikzpicture}
 \begin{center}
  \footnotesize{Debian's \texttt{key\_packages} are 84.5\% reproducible,
  but 435 packages (12.8\%) \\ 
  will still need to be fixed}
  \vfill
 \end{center}
\end{frame}


\begin{frame}
 \frametitle{Progress in the Debian bug tracker}
 \begin{tikzpicture}[remember picture]
  \node[shift={(-0.75\paperwidth, -0.3\paperheight)},at=(current page.south east)] {
    \includegraphics[height=0.65\paperheight]{images/stats_bugs_sin_ftbfs_state.png}
  };
 \end{tikzpicture}
 \begin{center}
  \footnotesize{As a rule, we file bugs with patches. \\
  There were very few exceptions.}
  \vfill
 \end{center}
\end{frame}

\begin{frame}
 \frametitle{What we did in Debian}

 \begin{itemize}
  \item Agreed on using a fixed build path: \texttt{/build/}
  \item Recording the build environment: \texttt{.buildinfo}
  \item \texttt{strip-nondeterminism}
  \item \texttt{diffoscope} (formerly \texttt{debbindiff})
  \item \texttt{SOURCE\_DATE\_EPOCH}
  \item \texttt{disorderfs}
  \item 1600+ patches: \texttt{dpkg}, \texttt{debhelper}, \texttt{sbuild}, …
  \item 4 packages modified to achive those 90.1\%
  \item …
 \end{itemize}
\end{frame}


\begin{frame}
 \frametitle{Tell the world \& collaborate}

 \begin{itemize}
  \item Weekly reports since May 2015
  \item<2-4> First Reproducible World Summit in December 2015 (Athens, Greece)
   \begin{itemize}
    \item<2-4> 40 people from 16 projects
    \item<2-4> \texttt{reproducible.debian.net} has become
  \texttt{tests.reproducible-builds.org}
  \item<3-4> another summit in second half 2016, somewhere in Europe
   \end{itemize}
  \item<4> 2 GSoC students in 2015, totally new contributors, totally rocking
  \item<4> 4 GSoC and Outreachy students in 2016
 \end{itemize}
\end{frame}

\begin{frame}
 \frametitle{Detour: Reproducible builds demand a defined build environment}
 \begin{itemize}
  \item …and being able to re-create this build environment is mandatory too.
  \item Without an \textit{sufficiently identical} build environment, reproducible builds will only
  happen by sheer luck.
  \item<2>{I've only verified for Debian so far… \texttt{koji}
  is designed for that too, Guix as well…}
\item<2> {I'd very much like to be corrected here, with tests.}
 \end{itemize}
\end{frame}


\begin{frame}
 \frametitle{Debian \texttt{.buildinfo} files}

 \begin{itemize}
  \item Aggregates in the same file:
   \begin{itemize}
    \item Sources (checksums)
    \item Generated binaries (checksums)
    \item Packages used to build (with specific version, checksums coming soon)
   \end{itemize}
  \item Can be later used to exactly recreate environment
  \item For Debian, all versions are available from \url{snapshot.debian.org}
 \end{itemize}
\end{frame}


\begin{frame}[fragile]
 \frametitle{Example \texttt{.buildinfo} file}

{\small
\begin{verbatim}
Format: 1.9
Build-Architecture: amd64
Source: txtorcon
Binary: python-txtorcon
Architecture: all
Version: 0.11.0-1
Build-Path: /build/txtorcon-0.11.0-1
Checksums-Sha256:
 a26549d9…7b 125910 python-txtorcon_0.11.0-1_all.deb
 28f6bcbe…69 2039 txtorcon_0.11.0-1.dsc
Build-Environment:
 base-files (= 8),
 base-passwd (= 3.5.37),
 bash (= 4.3-11+b1),
 …
\end{verbatim}
}
\end{frame}




\begin{frame}
 \frametitle{Blockers for Debian: \texttt{dpkg} and \texttt{dak}}
 \begin{itemize}
  \item \texttt{dpkg}
 \begin{itemize}
  \item \texttt{\#138409}: dpkg-dev: please add support for .buildinfo files
  \item \sout{\texttt{\#719844}: make compression of \{data,control\}.tar.gz deterministic}
  \item \texttt{\#759999}: set reproducible timestamps in \texttt{.deb} ar file headers
  \item \texttt{\#787980}: normalize file permissions when creating control.tar
  \item \texttt{\#719845}: make file order within {data,control}.tar.gz deterministic
 \end{itemize}
 \item \texttt{dak} (\texttt{ftp.debian.org})
 \begin{itemize}
  \item \texttt{\#763822}: please include \texttt{.buildinfo} files in the archive
 \end{itemize}
 \end{itemize}
\end{frame}

\begin{frame}
 \frametitle{\texttt{debian-policy}}

 \begin{itemize}
  \item Section 4.15: “Sources \textbf{must} build reproducible binaries.”
  \item<2-3> We hope this will happen after stretch (Debian 9) release
  \item<3> In 2016: “Sources \textbf{shall} build reproducible binaries.” ?
 \end{itemize}
\end{frame}

\begin{frame}
 \frametitle{Reminder / Summary}
 \begin{itemize}
  \item This is just a proof-of-concept, Debian is not 90\% reproducible,
  Debian is 0\% reproducible.
  \item Patches still need to be merged
  \item<2-4> I hope that Debian 9, "stretch", will be partially reproducible in a meaningful way
  \item<3-4> Debian \texttt{unstable} still needs changes to \texttt{dpkg} and
  \texttt{ftp.debian.org} (for keeping \texttt{.buildinfo} files)
  \item<4> what's beyond (rebuilding, \texttt{.buildinfo} file signing and
  distribution, user tools) mostly still needs \it{design} and code

 \end{itemize}
\end{frame}



\section{Status Non-Debian World}

\placelogofalse

\begin{frame}
 \frametitle{Status coreboot}
 \begin{itemize}
  \item \texttt{https://tests.r-b.org/coreboot}
  \item 99.2\% reproducible with \texttt{seabios} payload
  \item tests maintained by Alexander 'lynxis' Couzens
  \item unclear what the next steps are… they don't release binaries…
  \item needs involvement from coreboot developers
 \end{itemize}
 \begin{tikzpicture}[remember picture,overlay]
  \node[shift={(-0.13\paperwidth, 0.18\paperheight)},at=(current page.south east)] {
    \includegraphics[height=0.33\paperheight]{images/coreboot.png}
  };
 \end{tikzpicture}
\end{frame}

\begin{frame}
 \frametitle{Status OpenWrt}
 \begin{itemize}
  \item \texttt{https://tests.r-b.org/coreboot}
  \item selected images are 100\% reproducible and selected packages 99.7\%
  \item using 13 patches send upstream on January 25th
  \item tests maintained by Alexander 'lynxis' Couzens and Bryan Newbold
  \item recreating the build env: needs to checked in practice
  \item user verification tools: not yet
  \item next, once patches are merged: rebuilding released binaries?!
 \end{itemize}
 \begin{tikzpicture}[remember picture,overlay]
  \node[shift={(-0.16\paperwidth, 0.1\paperheight)},at=(current page.south east)] {
    \includegraphics[height=0.4\paperheight]{images/openwrt.png}
  };
 \end{tikzpicture}
\end{frame}

\begin{frame}
 \frametitle{Status NetBSD}
 \begin{itemize}
  \item \texttt{https://tests.r-b.org/netbsd}
  \item 21 (38.8\%) out of 54 built NetBSD files are reproducible
  \item tests maintained by Thomas 'wiz' Klausner and h01ger
  \item \texttt{MKREPRO=yes}
  \item \texttt{MK\_TIMESTAMP=\$SOURCE\_DATE\_EPOCH}
  \item recreating the build env: ?
  \item user verification tools: not yet
  \item next: ask Thomas :)
 \end{itemize}
 \begin{tikzpicture}[remember picture,overlay]
  \node[shift={(-0.13\paperwidth, 0.18\paperheight)},at=(current page.south east)] {
    \includegraphics[height=0.33\paperheight]{images/netbsd.png}
  };
 \end{tikzpicture}
\end{frame}

\begin{frame}
 \frametitle{Status FreeBSD}
 \begin{itemize}
  \item \texttt{https://tests.r-b.org/freebsd}
  \item base system not yet reproducible, but almost there
  \item 63\% of 15k ports were reproducible in 2013 already, their wiki says
  \item tests maintained by h01ger
  \item recreating the build env: ?
  \item user verification tools: not yet
  \item next: test ports?
 \end{itemize}
 \begin{tikzpicture}[remember picture,overlay]
  \node[shift={(-0.13\paperwidth, 0.2\paperheight)},at=(current page.south east)] {
    \includegraphics[height=0.33\paperheight]{images/freebsd.png}
  };
 \end{tikzpicture}
\end{frame}


\begin{frame}
 \frametitle{Status Fedora}
 \begin{itemize}
  \item \texttt{https://tests.r-b.org/fedora} (23)
  \item maintained by Dhiru Kholia and h01ger
  \item rpm repo available by Dhiru, but still \textbf{0\% reproducible}
  \item first patch for \texttt{rpm} merged
  \item rpm format includes build time and build host and signatures…
  \item recreating the build env: koji
  \item next: test 24+rawhide
  \item next: first reproducible rpm, use koji
  \item next: get more people involved
 \end{itemize}
 \begin{tikzpicture}[remember picture,overlay]
  \node[shift={(-0.13\paperwidth, 0.2\paperheight)},at=(current page.south east)] {
    \includegraphics[height=0.33\paperheight]{images/fedora.png}
  };
 \end{tikzpicture}
\end{frame}

\begin{frame}
 \frametitle{Status Arch Linux}
 \begin{itemize}
  \item \texttt{https://tests.r-b.org/archlinux}
  \item maintained by Levente 'anthraxx' Polyak and h01ger
  \item reproducible patches available for \texttt{pacman} by anthraxx
  \item recreating the build env: unaddressed
  \item next: use those patches, upstream them
 \end{itemize}
 \begin{tikzpicture}[remember picture,overlay]
  \node[shift={(-0.13\paperwidth, 0.2\paperheight)},at=(current page.south east)] {
    \includegraphics[height=0.33\paperheight]{images/archlinux.png}
  };
 \end{tikzpicture}
\end{frame}

\begin{frame}
 \frametitle{Status F-Droid}
 \begin{itemize}
  \item not yet: \texttt{https://tests.r-b.org/f-droid}
  \item maintained by Hans-Christoph Steiner and h01ger
  \item work has just begun…
 \end{itemize}
\end{frame}

\begin{frame}
 \frametitle{Unmentioned, with known activities}
 \begin{itemize}
\item Bitcoin, Tor,
\item Signal
\item OpenSUSE (could be tested easily…)
\item Ubuntu
\item Guix, NixOS
\item ElectroBSD
\item Qubes, TAILS, Subgraph OS
\item    commercial, propietary Software
\item ?
 \end{itemize}
\end{frame}

\begin{frame}
 \frametitle{Detour: what, reproducible commercial Software???}
 \begin{itemize}
\item Guess which
\item <2-3>   Microsoft Windows? (the source is available)
\item <2-3>   medical devices in your body?
\item <2-3>   arms?
\item <2-3>   critical infrastructure like in nuclear powerplants?
\item <2-3>   cars?
\item <3> Gambling machines!
 \end{itemize}
\end{frame}

\begin{frame}
 \frametitle{Unmentioned, unknown activities?}
 \begin{itemize}
\item  OpenBSD
\item  Gentoo (stage1)
 \item ?
\end{itemize}
\end{frame}


\placelogotrue

\section{Future work}




\begin{frame}
 \frametitle{Distributing \texttt{.buildinfo} files}
 \begin{itemize}
  \item Probably 100,000 new files per Debian suite; 50\% increase per suite
  \item Mirrors would not be happy, so should not go there
  \item<2> We'll need more files with detached signatures…
  \item<2>{Revoking signatures?}
  \item<2>{...}
 \end{itemize}
\end{frame}

\begin{frame}
 \frametitle{Rebuilders and sharing signed checksums}
 \begin{itemize}
  \item Almost no work has been done here yet.
  \item<2-3> Systematic, automatic rebuilds? 
  \item<3> Different projects, different solutions? 
 \end{itemize}
\end{frame}

\begin{frame}
 \frametitle{Rebuilders and sharing signed checksums, cont.}
 \begin{itemize}
  \item Individuelly signed checksums (think web of trust) could work in the
  Debian case (we have a gpg web of trust), but IMO won't scale.
  \item<2-3> { Another idea: rebuilders, run by large organisations
  (ACLU, CCC, CERN, Deutsche Bank, EDF, EON, Greenpeace, NASA, NSA, XYZ).}
  \item<3> Fedora rebuilds Debian, Debian rebuilds OpenSUSE, OpenSUSE rebuilds
  NetBSD, etc…
 \end{itemize}
\end{frame}


\begin{frame}
 \frametitle{Integration in user tools}
 \begin{itemize}
  \item "Do you really want to install this unreproducible software (y/N)"
  \item<2-4> "Do you want to build those packages which unconfirmed checksums,
  before installing? (Y/n)"
  \item<3-4>{ "How many signed checksums do you require to call a package
  'reproducible'?"}
  \item<4>{ "Which rebuilders do you want to trust?"}
 \end{itemize}
\end{frame}


\begin{frame}
 \frametitle{Summary}
 \begin{itemize}
  \item We've come a long way.
  \item We've made impressive progress.
  \item We're still not nearly where we want to be.
  \item<2-3> In fact, it's still fully not clear where we need to be going.
  \item<3> Keep up the great work!
  \item<3> Join the fun! There are many big and small things to do!
 \end{itemize}
\end{frame}


\section{Getting involved}

\begin{frame}
 \frametitle{As a software developer}
 \begin{itemize}
  \item Merge our patches
  \item<2> Stop using build dates
  \item<2> Use \texttt{SOURCE\_DATE\_EPOCH} instead
  \item<2> See \url{https://reproducible-builds.org/specs/}
 \end{itemize}
\end{frame}

\begin{frame}
 \frametitle{Getting involved - learning by doing}

 \begin{itemize}
  \item Test for yourself:
   \begin{itemize}
    \item Build something twice, run diffoscope on the results
    \begin{itemize}
     \item For better results use our “reproducible” repository, \texttt{pbuilder} and a custom config
    \end{itemize}
   \end{itemize}
  \item Docs on the web: \\
    \small{\url{https://reproducible-builds.org/docs/}} \\
    \small{\url{https://wiki.debian.org/ReproducibleBuilds/ExperimentalToolchain}}
  \item Ask for help on \texttt{\#debian-reproducible} or on mailing list
 \end{itemize}
\end{frame}

\begin{frame}
 \frametitle{Join the Reproducible builds team(s)!}

 \begin{itemize}
  \item Why?
   \begin{itemize}
    \item \heartsuit{}\heartsuit{}\heartsuit{} Lovely group of people \heartsuit{}\heartsuit{}\heartsuit{}
    \item Learn something new everyday
    \item Change the (software) world!
   \end{itemize}
  \item What do we do?
   \begin{itemize}
    \item Review packages
    \item Identify issues and document solutions
    \item \texttt{reproducible.d.n}, diffoscope, strip-nondeterminism
    \item Propose changes for toolchain
    \item Submit patches for individual packages
    \item Write more general documentation and talk to the world
   \end{itemize}
 \end{itemize}
\end{frame}


\section{Questions, comments, ideas?}


\begin{frame}
 \frametitle{Questions, comments, ideas?}

 \begin{itemize}
  \item<2-3> \url{https://reproducible-builds.org/docs}
  \item<2-3> \url{https://tests.reproducible-builds.org}
  \item<2-3> \texttt{\#reproducible-builds} on \texttt{irc.OFTC.net}
  \item<2-3> \small{and/or \texttt{\#debian-reproducible} too!}
  \item<2-3> \small\it{with IRC notification by \texttt{KGB-0}}
  \item<3> \url{https://lists.reproducible-builds.org}
  \item<3> \url{https://twitter.com/ReproBuild}
  \end{itemize}
\end{frame}


\begin{frame}
 \frametitle{Thanks to…! …and thank \textbf{you}, too!}

 \begin{itemize}
  \item
    \only<1>{Debian “Reproducible Builds” team \\
        {\small (you are just \textbf{so} awesome!)}}
    \only<2>{All “Reproducible Builds” teams \\
        {\small (you are just \textbf{so} awesome!)}}
  \item Linux Foundation and the Core Infrastructure Initiative
\end{itemize}

 \begin{center}
  \includegraphics[height=0.1\paperheight]{images/linux_foundation_logo.png}
  \hspace{0.1\paperwidth}
  \includegraphics[height=0.1\paperheight]{images/cii_logo.png}
 \end{center}

 \vfill
 \begin{center}
  \resizebox{0.9\textwidth}{!}{%
   \begin{tabular}{rl}
    \texttt{holger@debian.org} & \texttt{B8BF 5413 7B09 D35C F026} \\
                               & \texttt{FE9D 091A B856 069A AA1C}
   \end{tabular}
  }
 \end{center}
\end{frame}

\begin{frame}{}
\begin{textblock}{12}(2, 6)
    \tiny{
      Copyright \copyright{} 2014--2016 \\
         Holger Levsen \texttt{holger@layer-acht.org} and others.\\[3.0mm]
      Copyright of images included in this document are held by
      their respective owners.
      \\[3.0mm]
      This work is licensed under the \alert{Creative Commons
        Attribution-Share Alike 3.0} License.  To view a copy of this
      license, visit
      \url{http://creativecommons.org/licenses/by-sa/3.0/} or send a
      letter to Creative Commons, 171 Second Street, Suite 300, San
      Francisco, California, 94105, USA.
      \\[2.0mm]
      The source of this document is available from
      \url{https://anonscm.debian.org/git/reproducible/presentations.git}.
    }
  \end{textblock}
\end{frame}

\end{document}
