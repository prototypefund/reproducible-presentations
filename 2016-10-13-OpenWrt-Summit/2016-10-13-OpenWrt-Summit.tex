\documentclass[14pt,aspectratio=169]{beamer}
\setbeamertemplate{caption}[numbered]
\setbeamertemplate{caption label separator}{:}
\setbeamercolor{caption name}{fg=normal text.fg}
\usepackage{amssymb,amsmath}
\usepackage{ifxetex,ifluatex}
\usepackage{fixltx2e} % provides \textsubscript
\usepackage{lmodern}
\ifxetex
  \usepackage{fontspec,xltxtra,xunicode}
  \defaultfontfeatures{Mapping=tex-text,Scale=MatchLowercase}
  \newcommand{\euro}{€}
\else
  \ifluatex
    \usepackage{fontspec}
    \defaultfontfeatures{Mapping=tex-text,Scale=MatchLowercase}
    \newcommand{\euro}{€}
  \else
    \usepackage[T1]{fontenc}
    \usepackage[utf8]{inputenc}
      \fi
\fi
% use upquote if available, for straight quotes in verbatim environments
\IfFileExists{upquote.sty}{\usepackage{upquote}}{}
% use microtype if available
\IfFileExists{microtype.sty}{\usepackage{microtype}}{}
\PassOptionsToPackage{hyphens}{url}
\usepackage{hyperref}
\usepackage{ulem}

% Comment these out if you don't want a slide with just the
% part/section/subsection/subsubsection title:
\AtBeginPart{
  \let\insertpartnumber\relax
  \let\partname\relax
  \frame{\partpage}
}
\AtBeginSection{
  \let\insertsectionnumber\relax
  \let\sectionname\relax
  \begin{frame}[plain]
    \tableofcontents[currentsection]
  \end{frame}
}
\AtBeginSubsection{
  \let\insertsubsectionnumber\relax
  \let\subsectionname\relax
  \frame{\subsectionpage}
}

\setlength{\parindent}{0pt}
\setlength{\parskip}{6pt plus 2pt minus 1pt}
\setlength{\emergencystretch}{3em}  % prevent overfull lines
\setcounter{secnumdepth}{0}
% Thanks Richard Darst on how to get a nice Beamer theme.
% See http://rkd.zgib.net/wiki/DebianBeamerThemes

\usepackage{multicol}
\usepackage[absolute,overlay]{textpos}
\usepackage{tikz}
\usepackage{ctable}
\usetikzlibrary{positioning}

\usebackgroundtemplate{\includegraphics[width=\paperwidth]{images/swirl-lightest.pdf}}
\newif\ifplacelogo
\placelogotrue
\logo{\ifplacelogo\includegraphics[viewport=274 335 360 440,width=1cm]{images/openlogo-nd.pdf}\fi}

\definecolor{debianred}{rgb}{.780,.000,.211} % 199,0,54
\definecolor{debianblue}{rgb}{0,.208,.780} % 0,53,199
\definecolor{debianlightbackgroundblue}{rgb}{.941,.941,.957} % 240,240,244
\definecolor{debianbackgroundblue}{rgb}{.776,.784,.878} % 198,200,224

\usetheme{Boadilla}
\setbeamertemplate{navigation symbols}{}

\usecolortheme[named=debianbackgroundblue]{structure}
\setbeamercolor{normal text}{fg=black}
\setbeamercolor{titlelike}{fg=debianblue}
\setbeamercolor{sidebar}{fg=debianred,bg=debianbackgroundblue}

\setbeamercolor{palette sidebar primary}{fg=debianred}
\setbeamercolor{palette sidebar secondary}{fg=debianred}
\setbeamercolor{palette sidebar tertiary}{fg=debianred}
\setbeamercolor{palette sidebar quaternary}{fg=debianred}

\setbeamercolor{section in toc}{fg=debianred}
\setbeamercolor{subsection in toc}{parent=debianred}

\setbeamercolor{item}{fg=debianred}

\setbeamercolor{block title}{fg=debianblue}

\title[Reproducible builds everywhere]{Reproducible
builds everywhere \\ eg. in Debian, OpenWrt and LEDE}
\subtitle{Bit by bit identical binaries \\
from a given source}
\author[h01ger and lynxis]{%
   \texorpdfstring{
            \centering
            Alexander 'lynxis' Couzens\\
            Holger 'h01ger' Levsen
   }{h01ger and lynxis}}
\date[OpenWrt Summit, Berlin]{%
 OpenWrt Summit in Berlin, Germany\\
 \small{2016-10-13}}

\begin{document}

\begin{frame}[plain]
 \titlepage
\end{frame}

\begin{frame}
 \frametitle{about h01ger}

 \begin{itemize}
  \item \small{\texttt{B8BF 5413 7B09 D35C F026  FE9D 091A B856 069A AA1C}}
  \item Debian user since 1995
  \item Debian contributor since 2001
  \item OpenWrt user since 2006
  \item Debian developer since 2007
  \item DebConf organizer,
  founded the DebConf video team
   \begin{itemize}
    \item \texttt{http://video.debian.net}
   \end{itemize}
 \item Debian-Edu (Debian for education)
  \item Debian QA (quality assurance)
  \begin{itemize}
   \item \texttt{https://piuparts.debian.org}
   \item \texttt{https://jenkins.debian.net} (~1200 jobs continously testing Debian)
  \end{itemize}
  \item Debian Reproducible builds team member
  \begin{itemize}
   \item since April 2015 funded by the Linux Foundation
 \end{itemize}
 \end{itemize}
\end{frame}

\placelogofalse

\begin{frame}
 \frametitle{about lynxis}

 \begin{itemize}
  \item \small{\texttt{390D CF78 8BF9 AA50 4F8F  F1E2 C29E 9DA6 A0DF 8604}}
  \item Debian user since 2003
  \item OpenWrt user since 2006
  \item LEDE founding member
  \item coreboot hacker
  \item tests.reproducible-builds.org contributor
  \item CCC member
 \end{itemize}
 \begin{tikzpicture}[remember picture,overlay]
  \node[shift={(-0.08\paperwidth, 0.12\paperheight)},at=(current page.south east)] {
    \includegraphics[height=0.2\paperheight]{images/lede.png}
  };
 \end{tikzpicture}
 \begin{tikzpicture}[remember picture,overlay]
  \node[shift={(-0.85\paperwidth, 0.1\paperheight)},at=(current page.south east)] {
    \includegraphics[height=0.4\paperheight]{images/openwrt.png}
  };
 \end{tikzpicture}
\end{frame}


\begin{frame}
 \frametitle{about OpenWrt and LEDE}

 \begin{itemize}
  \item In this talk we'll ignore the distinction between the two:
  \item when we say "OpenWrt" me mean "LEDE and OpenWrt",
  \item when we say "LEDE" me mean "OpenWrt and LEDE",
  \item when we say "OpenWrt and LEDE" we mean "LEDE and OpenWrt".
  \item<2> They are two projects though and when there are differences we'll
  mention them.
 \end{itemize}
 \begin{tikzpicture}[remember picture,overlay]
  \node[shift={(-0.14\paperwidth, 0.1\paperheight)},at=(current page.south east)] {
    \includegraphics[height=0.4\paperheight]{images/openwrt.png}
  };
 \end{tikzpicture}
 \begin{tikzpicture}[remember picture,overlay]
  \node[shift={(-0.93\paperwidth, 0.11\paperheight)},at=(current page.south east)] {
    \includegraphics[height=0.2\paperheight]{images/lede.png}
  };
 \end{tikzpicture}

\end{frame}


\begin{frame}
 \frametitle{Who are you?}
 \begin{itemize}
  \item<2-4> Seen a talk about reproducible builds?
  \item<3-4> Contributed to the efford?
  \item<4> Uses Debian or a Debian based system?
 \end{itemize}
\end{frame}

\placelogotrue

\begin{frame}
 \frametitle{Debian reproducible builds team}
 \begin{center}
  \begin{columns}
   \footnotesize
   \column{.30\linewidth}
    {akira} \\
    {Alexis Bienvenüe} \\
    {Andrew Ayer} \\
    {Asheesh Laroia} \\
    {Ceridwen} \\
    {Chris Lamb} \\
    {Chris West} \\
    {Christoph Berg} \\
    {Daniel Kahn Gillmor} \\
    {Daniel Shahaf} \\
    David Suarez \\
    {Dhole} \\
    Drew Fisher \\
    Emmanuel Bourg \\
    Emanuel Bronshtein \\
    Esa Peuha \\
    \column{.30\linewidth}
    {Fabian Wolff} \\
    {Guillem Jover} \\
    Hans-Christoph Steiner \\
    {Helmut Grohne} \\
    \only<1>{Holger Levsen}\only<2>{{\color{debianred} Holger Levsen}} \\
    HW42 \\
    Intrigeri \\
    {Jelmer Vernooij} \\
    {josch} \\
    Juan Picca \\
    {Lunar} \\
    Mathieu Bridon \\
    {Mattia Rizzolo} \\
    Nicolas Boulenguez \\
    {Niels Thykier} \\
    Niko Tyni \\
   \column{.30\linewidth}
    {Paul Wise} \\
    Peter De Wachter \\
    Philip Rinn \\
    {Reiner Herrmann} \\
    {Santiago Vila} \\
    {Sascha Steinbiss} \\
    {Satyam Zode} \\
    {Scarlett Clark} \\
    {Stefano Rivera} \\
    {Stéphane Glondu} \\
    {Steven Chamberlain} \\
    Tom Fitzhenry \\
    {Valerie Young} \\
    Valentin Lorentz \\
    {Wookey} \\
    {Ximin Luo} \\
  \end{columns}
 \end{center}
\end{frame}


\begin{frame}
 \frametitle{jenkins.debian.net.git contributors}
 \begin{center}
  \begin{columns}
   \small
   \column{.46\linewidth}
    {akira} \\
    \only<1>{Alexander Couzens}\only<2>{{\color{debianred} Alexander Couzens}} \\
    \only<1>{Levente 'anthraxx' Polyak}\only<2>{{\color{debianred} Levente 'anthraxx' Polyak}} \\
    {Antonio Terceiro} \\
    {Axel Beckert} \\
    \only<1>{Bryan Newbold}\only<2>{{\color{debianred} Bryan Newbold}} \\
    {Chris Lamb} \\
    {Daniel Kahn Gillmor} \\
    {Gabriele Giacone} \\
    \only<1>{Hans-Christoph Steiner}\only<2>{{\color{debianred} Hans-Christoph Steiner}} \\
    Helmut Grohne \\
    \only<1>{Holger Levsen}\only<2>{{\color{debianred} Holger Levsen}} \\
    \only<1>{HW42}\only<2>{{\color{debianred} HW42}} \\
    {James McCoy} \\
    {Joachim Breitner} \\
   \column{.46\linewidth}
    {Johannes 'josch' Schauer} \\
    {Jérémy Bobbio} \\
    {Mattia Rizzolo} \\
    {Niels Thykier} \\
    {Paul Wise} \\
    {Petter Reinholdtsen} \\
    {Philip Hands} \\
    \only<1>{Reiner Herrmann}\only<2>{{\color{debianred} Reiner Herrmann}} \\
    {Samuel Thibault} \\
    {Steven Chamberlain} \\
    {Tails developers} \\
    {Ulrike Uhlig} \\
    {Wolfgang Schweer} \\
    {Wouter Verhelst} \\
  \end{columns}
 \end{center}
\end{frame}

\placelogofalse

\begin{frame}
 \frametitle{OpenWrt / LEDE reproducible builds contributors}
 \begin{center}
  \begin{columns}
  \column{.46\linewidth}
    {Alexander Couzens} \\
    {Bryan Newbold} \\
    {Dirk Neukirchen} \\
    {Felix Fietkau} \\
    {Jonas Gorski} \\
    {Jo-Philipp Wich} \\
    {Nathan Hintz} \\
    {Reiner Herrmann} \\
  \end{columns}
 \end{center}
 \begin{tikzpicture}[remember picture,overlay]
  \node[shift={(-0.08\paperwidth, 0.12\paperheight)},at=(current page.south east)] {
    \includegraphics[height=0.2\paperheight]{images/lede.png}
  };
 \end{tikzpicture}
 \begin{tikzpicture}[remember picture,overlay]
  \node[shift={(-0.85\paperwidth, 0.1\paperheight)},at=(current page.south east)] {
    \includegraphics[height=0.4\paperheight]{images/openwrt.png}
  };
 \end{tikzpicture}
\end{frame}

\section{Motivation}

\begin{frame}
 \frametitle{The problem}

 \begin{center}
  \includegraphics[width=0.7\textwidth]{images/31c3.png}

  Available on \url{media.ccc.de}, 31c3
 \end{center}
\end{frame}

\begin{frame}[fragile]
 \frametitle{A few examples from that 31c3 talk}
 \begin{itemize}
  \item CVE-2002-0083: remote root exploit in \texttt{sshd}, a single bit difference in the binary
  \item<2-5> 31c3 talk had a live demo with a kernel module modifying source code in memory only
  \item<3-5> How can you be sure what's running on your machine or on a build
  daemon network connected to the net? Do you ever leave your computers physically alone?
  \item<4-5> Huge financial incentives to crack developer machines or a projects
  build infrastructure…
 \end{itemize}
\end{frame}

\begin{frame}[fragile]
 \frametitle{Another example from real life}

 At a CIA conference in 2012:
 \begin{center}
  \includegraphics[width=0.8\textwidth]{images/strawhorse.png}

  {\footnotesize
  \url{firstlook.org/theintercept/2015/03/10/ispy-cia-campaign-steal-apples-secrets/}
  }
 \end{center}
\end{frame}


\begin{frame}
 \frametitle{The solution}

 \begin{center}
 \Large{
 Promise that anyone can always generate
 identical binary packages
 from a given source}
\end{center}
\end{frame}


\begin{frame}
 \frametitle{The solution}

 \begin{center}
 We call this:

 \Huge{ “Reproducible builds” }
 \end{center}
\end{frame}

\placelogotrue

\begin{frame}
 \frametitle{Debian demo (skipped)}
 \begin{itemize}
 \item Build a package 5 times, get 5 .debs with different checksums
 \item Build a package 5 times, get 5 .debs with the same checksum\\
 \only<2>{Yes, it's really this simple.}
 \end{itemize}
% show this once running in plain sid,
% and then in sid with our modified toolchain.
%
% prepare demo:
% mkdir demo ; cd demo ; apt-get source giftrans
%
% do demo:
% PTH=$(mktemp -d); OPTH=$PWD; P=giftrans; cp ${P}_* $PTH/; cd $PTH ;
%   dpkg-source -x ${P}*.dsc ; for X in 1 2 3 4 5 ; do (cd ${P}-*/;
%   dpkg-buildpackage -b -uc -us); mkdir -p .$X ; cp $P_*.deb .$X; done ; rm
%   *.deb ; echo; sha1sum *dsc *z .*/*.deb | grep -v giftrans-dbgsym ; cd - ;
% rm -r $PTH
\end{frame}

\placelogofalse

\begin{frame}[plain]
\begin{center}
 \Huge{This should become the \textbf{norm}.}

 \visible<2>{\small{ We want to change the meaning of "free software":

  it's only free software if it's reproducible!}}
\end{center}
\end{frame}

\begin{frame}[fragile]
 \frametitle{More benefits than "just" security…}
 \begin{itemize}
  \item smaller deltas, thus faster updates possible
  \item in Debian: lots of QA benefits
  \item Google does reproducible builds, to save money
  \item …
 \end{itemize}
\end{frame}


\section{Common ressources}

\begin{frame}
 \frametitle{reproducible-builds.org}

 \begin{itemize}
  \item \texttt{https://reproducible-builds.org}
  \item git repositories, IRC channels, mailinglists, webspace
 \end{itemize}
 \begin{center}
 \includegraphics[width=0.7\textwidth]{images/rbwww1.png}
 \end{center}
\end{frame}


{
\usebackgroundtemplate{%
 \begin{tikzpicture}[remember picture,overlay]%
  \node[shift={(-0.1\paperwidth, 0.15\paperheight)},at=(current page.south east)] {
    \includegraphics[width=0.2\paperwidth]{images/diffoscope_logo.png}
  };
 \end{tikzpicture}%
}

\begin{frame}{diffoscope}
 \frametitle{Debugging problems: \texttt{https://try.diffoscope.org}}

 \begin{itemize}
  \item Examines differences \textbf{in depth}.
  \item Outputs HTML or plain text with human readable differences.
  \item Recursively unpacks archives, uncompresses PDFs, disassembles
  binaries, unpacks Gettext files, …
  \item Easy to extend to new file formats.
  \item Falls back to binary comparison.
  \item Available from \texttt{git}, PyPI, Debian, \\
   Arch Linux, Guix, Homebrew. Works on BSD.
  \item Maintainers in other distros wanted.
  \item \url{https://diffoscope.org/}
 \end{itemize}
\end{frame}


\begin{frame}
 \frametitle{\texttt{diffoscope} example (HTML output)}
 \begin{tikzpicture}[remember picture]
  \node[at=(current page.center)] {
   \includegraphics[width=0.9\paperwidth]{images/diffoscope_example_html.png}
  };
 \end{tikzpicture}
\end{frame}


\begin{frame}
 \frametitle{\texttt{diffoscope} is "just" for debugging}

 \begin{itemize}
  \item Reminder: \texttt{diffoscope} is for \textbf{debugging}
  \item "reproducible" according to our definition means: \textbf{bit by bit
  identical}. So the tools for testing whether something is reproducible are
  either \texttt{diff} or \texttt{sha256sum}!
 \end{itemize}
\end{frame}

}

\placelogotrue


\begin{frame}
 \frametitle{tests.reproducible-builds.org}

 \begin{itemize}
  \item Continuously testing Debian \texttt{testing}, \texttt{unstable} and
  \texttt{experimental}
  \item Also testing: coreboot, OpenWrt, LEDE, NetBSD, FreeBSD,
  Arch Linux, Fedora and soon F-Droid too
  \item 8-12 \texttt{amd64} nodes, 150 cores and soon 500 GB RAM - thanks to
  Profitbricks.com!
  \item 22 \texttt{armhf} nodes, 98 cores and 53 GB RAM
  \item 329 jenkins jobs running on jenkins.debian.net
  \item 43 scripts in Python and Bash, 283 lines of code in average
  \item 37 contributors for \texttt{jenkins.debian.net.git}
 \end{itemize}
\end{frame}


\begin{frame}[fragile]
 \frametitle{Variations (when testing Debian)}

 \begin{center}
  \begin{table}
   \resizebox{0.95\textwidth}{!}{%
    \begin{tabular}{l|ll}
\textbf{variation} & \textbf{first build} & \textbf{second build} \\
\hline
hostname & \texttt{jenkins} & \texttt{i-capture-the-hostname} \\
domainname & \texttt{debian.net} & \texttt{i-capture-the-domainname} \\
\texttt{env TZ} & \texttt{GMT+12} & \texttt{GMT-14} \\
\texttt{env LANG} & \texttt{C} & \texttt{fr\_CH.UTF-8} \\
\texttt{env LC\_ALL} & not set & \texttt{fr\_CH.UTF-8} \\
\texttt{env USER} & \texttt{pbuilder1} & \texttt{pbuilder2} \\
uid & \texttt{1111} & \texttt{2222} \\
gid & \texttt{1111} & \texttt{2222} \\
UTS namespace & shared with the host & \textit{modified using \texttt{/usr/bin/unshare --uts}} \\
kernel version & Linux 3.16 or 4.X & on amd64 always varied, on armhf
sometimes \\
umask & 0022 & 0002 \\
CPU type & \multicolumn{2}{l}{varied on i386} \\
 & on armhf varied a bit, not on amd64 & \\
filesystem & \multicolumn{2}{l}{same for both builds on amd64: (\texttt{tmpfs}), on armhf \texttt{ext3/4}} \\
 & & \textit{(and we have} \texttt{disorderfs}\textit{, but the code is disabled)} \\
year, month, date & \multicolumn{2}{l}{on amd64: 398 days variation, on armhf not yet} \\
hour, minute & \multicolumn{2}{l}{hour is usually the same… usually, the minute differs… } \\
\textit{everything else} & \multicolumn{2}{l}{\textit{is likely the same…}}
    \end{tabular}
   }
  \end{table}
 \end{center}
\end{frame}

\placelogofalse

\begin{frame}
 \frametitle{Common problems}

 \begin{itemize}
  \item time stamps
  \item timezones
  \item locales
  \item build paths
  \item everything else (seperated into known issues and the blurry rest)
 \end{itemize}
\end{frame}

\begin{frame}
 \frametitle{Documentation about common problems}
 \begin{itemize}
  \item \texttt{https://reproducible-builds.org/docs}
  \item Lunar's talk from CCCamp 2015 also on
  \texttt{https://media.ccc.de}
 \begin{tikzpicture}[remember picture]
  \node[shift={(-1.05\paperwidth, -0.3\paperheight)},at=(current page.south east)] {
    \includegraphics[width=0.83\textwidth]{images/cccamp2015_lunar_random.png}
  };
 \end{tikzpicture}
 \end{itemize}
\end{frame}


\begin{frame}
 \frametitle{\texttt{SOURCE\_DATE\_EPOCH}}

 \begin{itemize}
  \item Build date (timestamps) usually not useful for the user
  \item \texttt{SOURCE\_DATE\_EPOCH} is defined as the last modification of
  the source, since the epoch (1970-01-01)
  \item can be used instead of current date
  \item can also be used for random seeds etc.
  \item in Debian, set from the latest \texttt{debian/changelog} entry
  \item can be set to the latest git commit too or the latest file
  modification date
 \end{itemize}
\end{frame}

\begin{frame}
 \frametitle{\texttt{SOURCE\_DATE\_EPOCH}}

 \begin{itemize}
  \item \texttt{SOURCE\_DATE\_EPOCH} spec available:
  \item \texttt{https://reproducible-builds.org/specs/}
  \item many upstreams support it already
  \item has been adopted by other distributions
  (OpenWrt, LEDE, NetBSD, FreeBSD, Arch Linux, coreboot, Guix, …) and many many
  upstreams (GCC, dpkg, rpm, mkisofs, ghostscript, libxslt, sphinx,
  texlive-bin, …)
 \end{itemize}
\end{frame}


\placelogotrue

\section{Status Debian}

\begin{frame}
 \frametitle{Progress in Debian \texttt{testing} ("stretch")}
 \begin{tikzpicture}[remember picture]
  \node[shift={(-0.5\paperwidth, \paperheight)},at=(current page.south east)] {
    \includegraphics[height=0.65\paperheight]{images/stats_pkg_state_testing.png}
  };
 \end{tikzpicture}
 \begin{center}
  \footnotesize{21,527 (91.2\%) out of 23,597 source packages are reproducible \\
    in our test framework on \texttt{amd64}}
  \vfill
 \end{center}
\end{frame}

\begin{frame}
 \frametitle{Progress in Debian \texttt{unstable}}
 \begin{tikzpicture}[remember picture]
  \node[shift={(-0.5\paperwidth, \paperheight)},at=(current page.south east)] {
    \includegraphics[height=0.65\paperheight]{images/stats_pkg_state_unstable.png}
  };
 \end{tikzpicture}
 \begin{center}
  \footnotesize{18,898 (75.8\%) out of 24,931 source packages are reproducible \\
    in our test framework on \texttt{amd64}} (difference due to build path variations)
  \vfill
 \end{center}
\end{frame}


\begin{frame}
 \frametitle{Progress in the Debian bug tracker}
 \begin{tikzpicture}[remember picture]
  \node[shift={(-0.5\paperwidth, \paperheight)},at=(current page.south east)] {
    \includegraphics[height=0.65\paperheight]{images/stats_bugs_sin_ftbfs_state.png}
  };
 \end{tikzpicture}
 \begin{center}
  \footnotesize{As a rule, we file bugs with patches. \\
  There are very few exceptions.}
  \vfill
 \end{center}
\end{frame}

\begin{frame}
 \frametitle{Details on tests.reproducible-builds.org}

 \begin{itemize}
  \item \url{https://reproducible.debian.net/$src}
  \item 43 package sets 
  \item 250 categorised distinct issues
  \item 6,944 notes
  \item 1,894 unreproducible packages in \texttt{stretch} (testing), but only
  177 without a note (5,777 in \texttt{unstable} but also only 277 without a
  note)
  \item maintained in \texttt{notes.git} by 47 contributors
  \item currently Debian only, but cross distro notes are planned
 \end{itemize}
\end{frame}


\begin{frame}
 \frametitle{Summary / What's left to do}
 \begin{itemize}
  \item This is a proof-of-concept, Debian is neither 91.2\% reproducible nor
  75.8\%. (and 10\% > 2,300 sources packages!)
  \item<2-4> All our required changes are finally in Debian now, except
  \texttt{dpkg} and \texttt{.buildinfo} file support on the archive side.
  \item<3-4> We hope that Debian 9, "stretch", will be partially reproducible
  in a meaningful way, in 2017.
  \item<3-4> What's beyond (rebuilding, \texttt{.buildinfo} file handling, user
  tools) still needs \it{design} and code.
  \item<4> Will Debian 10, "buster", be 100\% reproducible?
 \end{itemize}
\end{frame}





\begin{frame}
 \frametitle{Tell the world \& collaborate}

 \begin{itemize}
  \item "We don't care about Debian (only), we care about free and open
   source software."
  \item<2-4> Weekly reports since May 2015
  \item<3-4> First Reproducible World Summit in December 2015 (Athens, Greece)
   \begin{itemize}
    \item<3-4> 40 people from 16 projects
    \item<3-4> \texttt{reproducible.debian.net} has become \texttt{tests.reproducible-builds.org}
   \end{itemize}
    \item<4> Second Reproducible World Summit in December 2016 in Berlin
   \begin{itemize}
    \item<4> Talk to h01ger if you want to attend.
   \end{itemize}
 \end{itemize}
\end{frame}



\section{Status Non-Debian World}

\placelogofalse

\begin{frame}
 \frametitle{Skipping some…}
 \begin{itemize}
  \item \texttt{https://tests.r-b.org/coreboot}
  \item \texttt{https://tests.r-b.org/netbsd}
  \item \texttt{https://tests.r-b.org/freebsd}
  \item paused: \texttt{https://tests.r-b.org/archlinux}
  \item paused: \texttt{https://tests.r-b.org/fedora}
  \item not yet: \texttt{https://tests.r-b.org/f-droid}
 \end{itemize}
\end{frame}


\begin{frame}
 \frametitle{Skipping some more…}
 \begin{itemize}
\item Bitcoin (2011)
\item Tor (2013)
\item NixOS, Guix, ElectroBSD
\item Qubes, Tails
\item very few commercial, propietary software (\only<1>{guess
where}\only<2>{gamblingmachines}!)
\item ?
 \end{itemize}
\end{frame}


\begin{frame}
 \frametitle{OpenWrt and LEDE tested for reproducible builds}
 \begin{itemize}
  \item \texttt{https://tests.r-b.org/openwrt}
  \item \texttt{https://tests.r-b.org/lede}
  \item maintained by lynxis, deki and h01ger
  \item reproducible\_(openwrt|lede).sh scripts in \texttt{jenkins.debian.net.git}
  \item 1,073/1,089 packages and 1/12 (OpenWrt/LEDE) images tested each week
  \item variations: TZ, LANG, LC\_ALL, PATH, (umask), make -j, linux64 --uname-2.6, CAPTURE\_ENVIRONMENT
 \end{itemize}

 \begin{tikzpicture}[remember picture,overlay]
  \node[shift={(-0.14\paperwidth, 0.1\paperheight)},at=(current page.south east)] {
    \includegraphics[height=0.4\paperheight]{images/openwrt.png}
  };
 \end{tikzpicture}
 \begin{tikzpicture}[remember picture,overlay]
  \node[shift={(-0.93\paperwidth, 0.11\paperheight)},at=(current page.south east)] {
    \includegraphics[height=0.2\paperheight]{images/lede.png}
  };
 \end{tikzpicture}
\end{frame}

\begin{frame}
 \frametitle{TODO for tests.r-b.org/(openwrt|lede)}
 \begin{itemize}
  \item we should add more variations (date, time, hostname, domain, use disorderfs, CPU type,
  kernel, USER, HOME, SHELL, the base system)
  \item<2-3> we want to change what we are building to make \textbf{you} look at these
  pages eve
  \item<2-3> we could build other branches too…
  \item<2-3> we could build OpenWrt + LEDE at least every day, thanks again to
  Profitbricks.com
  \item<3> collaboration: \texttt{notes.git} shall get multi-project support
 \end{itemize}
 \begin{tikzpicture}[remember picture,overlay]
  \node[shift={(-0.08\paperwidth, 0.12\paperheight)},at=(current page.south east)] {
    \includegraphics[height=0.2\paperheight]{images/lede.png}
  };
 \end{tikzpicture}
 \begin{tikzpicture}[remember picture,overlay]
  \node[shift={(-0.85\paperwidth, 0.1\paperheight)},at=(current page.south east)] {
    \includegraphics[height=0.4\paperheight]{images/openwrt.png}
  };
 \end{tikzpicture}
\end{frame}

\begin{frame}
 \frametitle{TODO: design \texttt{.buildinfo} files for OpenWrt and LEDE}
 \begin{itemize}
  \item rfc822 format
  \item needs to define the environment
  \item needs to define the sources (input)
  \item needs to define the binaries (output)
  \item<2-3> Debian has only .deb files as output, while OpenWrt/LEDE have packages
  and images…
  \item<3> OTOH the OpenWrt/LEDE build environment base is not defined in the
  OpenWrt/LEDE git tag… IOW: you can build on Debian, Fedora or even on MacOS…
 \end{itemize}
 \begin{tikzpicture}[remember picture,overlay]
  \node[shift={(-0.14\paperwidth, 0.1\paperheight)},at=(current page.south east)] {
    \includegraphics[height=0.4\paperheight]{images/openwrt.png}
  };
 \end{tikzpicture}
 \begin{tikzpicture}[remember picture,overlay]
  \node[shift={(-0.93\paperwidth, 0.11\paperheight)},at=(current page.south east)] {
    \includegraphics[height=0.2\paperheight]{images/lede.png}
  };
 \end{tikzpicture}
\end{frame}


\section{Future work}

\begin{frame}
 \frametitle{Rebuilds and sharing signed checksums}
 \begin{itemize}
  \item Almost no work has been done here yet. We are just at the first step:
  being able to rebuild reproducibly…
  \item Different projects, different solutions?
 \begin{itemize}
  \item<2> something like \texttt{.buildinfo} files (defining the environment,
  the input and the output(s)) will be needed everywhere, but so far we only
  have them for Debian…
 \end{itemize}
 \end{itemize}
\end{frame}

\begin{frame}
 \frametitle{Rebuilders and sharing signed checksums, cont.}
 \begin{itemize}
  \item Individuelly signed checksums (think web of trust) could work in the
  Debian case (we have a gpg web of trust), but IMO won't scale.
  \item { Another idea: rebuilders, run by large organisations
  (ACLU, CCC, CERN, Deutsche Bank, EDF, EON, Greenpeace, NASA, NSA, XYZ).}
  \item Fedora rebuilds Debian, Debian rebuilds OpenSUSE, OpenSUSE rebuilds
  NetBSD, etc…
  \item Big customers could just rebuild everything themselves.
 \end{itemize}
\end{frame}


\begin{frame}
 \frametitle{Integration in user tools}
 \begin{itemize}
  \item "Do you really want to install this unreproducible software (y/N)"
  \item<2-3> "Do you want to build those packages which unconfirmed checksums,
  before installing? (Y/n)"
  \item<3>{ "How many signed checksums do you require to call a package
  'reproducible'?" - and whom do you trust?}
 \end{itemize}
\end{frame}


\section{Getting involved}

\begin{frame}
 \frametitle{As a software developer}
 \begin{itemize}
  \item Stop using build dates
  \item Use \texttt{SOURCE\_DATE\_EPOCH} instead
  \item See \url{https://reproducible-builds.org/specs/}
 \end{itemize}
\end{frame}

\begin{frame}
 \frametitle{Getting involved - learning by doing}

 \begin{itemize}
  \item Test for yourself:
   \begin{itemize}
    \item Build something twice, run diffoscope on the results
   \end{itemize}
  \item Docs on the web: \\
    \small{\url{https://reproducible-builds.org/docs/}} \\
  \item Ask for help on \texttt{\#reproducible-builds} or on mailing list
 \end{itemize}
\end{frame}


\begin{frame}
 \frametitle{Form your reproducible builds team!}

 \begin{itemize}
  \item Why?
   \begin{itemize}
    \item Every distribution should be reproducible!
    \item Learn something new everyday
    \item Change the (software) world!
    \item \texttt{https://tests.reproducible-builds.org/openwrt} needs \textbf{your} help
    \item \texttt{https://tests.reproducible-builds.org/lede} needs \textbf{your} help
   \end{itemize}
  \item How to get started?
   \begin{itemize}
    \item Talk to lynxis or h01ger here or talk to us on IRC or via mail.
    \item RTFM, there is lots of documentation
    \item Experiment - learning by doing
    \item Use+debug existing patches to \texttt{rpm-build}
   \end{itemize}
 \end{itemize}
\end{frame}

\section{Questions, comments, ideas?}


\begin{frame}
 \frametitle{Questions, comments, ideas?}

 \begin{itemize}
  \item \url{https://reproducible-builds.org/}
  \item \texttt{\#reproducible-builds} on \texttt{irc.OFTC.net}
  \item \url{https://lists.reproducible-builds.org}
  \item twitter: @ReproBuild
  \item<2> Mike and Seth's talk from 31c3 about motivations
  \item<2> Lunar's talk about fixing reproducible issues from CCCamp 15
  \item<2> h01ger's talk "the Reproducible builds ecosystem" from FOSDEM 16
  \end{itemize}
\end{frame}

\placelogotrue

\begin{frame}
 \frametitle{Thanks to…! …and thank \textbf{you}, too!}

 \begin{itemize}
  \item
    {All “Reproducible Builds” contributors \\
        {\small (you are just \textbf{so} awesome!)}}
  \item OpenWrt Summit and ELCE
\end{itemize}

 \begin{center}
  \includegraphics[height=0.1\paperheight]{images/linux_foundation_logo.png}
  \hspace{0.1\paperwidth}
  \includegraphics[height=0.1\paperheight]{images/cii_logo.png}
  \hspace{0.1\paperwidth}
  \includegraphics[height=0.1\paperheight]{images/profitbricks_logo.png}
 \end{center}

 \vfill
 \begin{center}
  \resizebox{0.9\textwidth}{!}{%
   \begin{tabular}{rl}
    \texttt{holger@debian.org} & \texttt{B8BF 5413 7B09 D35C F026} \\
                               & \texttt{FE9D 091A B856 069A AA1C} \\
    \texttt{lynxis@fe80.eu} & \texttt{390D CF78 8BF9 AA50 4F8F} \\
                               & \texttt{F1E2 C29E 9DA6 A0DF 8604}
\end{tabular}
  }
 \end{center}
\end{frame}


\begin{frame}{}
\begin{textblock}{12}(2, 6)
    \tiny{
      Copyright \copyright{} 2014--2016 \\
         Holger Levsen \texttt{holger@layer-acht.org} and others.\\[3.0mm]
      Copyright of images included in this document are held by
      their respective owners.
      \\[3.0mm]
      This work is licensed under the \alert{Creative Commons
        Attribution-Share Alike 3.0} License.  To view a copy of this
      license, visit
      \url{http://creativecommons.org/licenses/by-sa/3.0/} or send a
      letter to Creative Commons, 171 Second Street, Suite 300, San
      Francisco, California, 94105, USA.
      \\[2.0mm]
      % Give a link to the 'Transparent Copy', as per Section 3 of the GFDL.
      The source of this document is available from
      \url{https://anonscm.debian.org/git/reproducible/presentations.git}.
    }
  \end{textblock}
\end{frame}

\end{document}
